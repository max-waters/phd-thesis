\section{Experimental Evaluation}\label{sec:pp-evaluation}
%
The notions of optimality discussed above reflect the central challenge of partial plans, which is to find a plan that is flexible while still being tractable enough to be useful at execution time.
But, this theoretical examination cannot reveal whether, in practice, the strict tractability requirements are so limiting that they render the plans unable to provide any flexibility.
Section~\ref{sec:pp-cs-plans} demonstrated that finding a minimally constrained, tractable \CLP is feasible. 
However, the question remains whether \MKTR can generate (or indeed, whether there even exist) partial plans that \myi provide more flexibility than the \POP{}s produced by simpler plan de/reordering techniques, and \myii have an instantiation cost within the bounds of what is suitable for real-time decision-making. 

To address this question, \MKTR has been assessed over all previous \IPC \STRIPS domains.
As a baseline, \MKTR was compared with the \emph{explanation-based order generalisation} deordering technique of \citet{Kambhampati:2004:ExplBasedGeneralisation} (\EOG, Definitions~\ref{def:eog-valid-struct} and~\ref{def:eog}). 
As discussed in Section~\ref{sec:kk-ebg}, \EOG is a greedy \POP deordering algorithm that uses the input plan's causal structure as a deordering heuristic.
Despite this lack of optimality guarantee, Chapter~\ref{chap:pop-maxsat} confirmed the results of \citet{Muise2016-PopMaxSAT} that in all cases, \EOG finds a minimal deorder (Definition~\ref{def:opt-de-reorder}) of the input plan.

Results show that in all test cases, \MKTR produces a partial plan of quadratic ``complexity'' that encapsulates all linearisations of the \POP found by \EOG. 
While certain planning domains resist any kind of relaxation, on average \MKTR finds a partial plan with $163.2\%$ more instantiations than \EOG, and increasing the maximum treewidth from two to five increases this to $302.3\%$ with little change in instantiation time.
%
%%%%%%%%%%%%%%%%%%%%%%%%%%%%%%%%%%%%%%%%%%%%%%%%%%%%%%%%%%%%%%%%%%%%%%%%%%%%%%%%
\subsection{Experimental Setup} 
%%%%%%%%%%%%%%%%%%%%%%%%%%%%%%%%%%%%%%%%%%%%%%%%%%%%%%%%%%%%%%%%%%%%%%%%%%%%%%%%
%
As in Chapter~\ref{chap:pop-maxsat}, test cases (i.e., input plans) were generated by finding plans for all first-order \IPC \STRIPS planning instances.
To ensure a variety of plans, three planners of distinct "types" were used: the novelty-driven best-first search planner \dbfws~\cite{Lipovetzky2017:BFWS}, the heuristic forward-search planner \lama~\cite{RichterW:LAMA} and the \SAT planner \madagascar~\cite{Rinatnnen2010:M}. 

Each plan was relaxed with \MKTR under eight different configurations. 
The \emph{Minimise Threats} and \emph{Relax Producers} policies were tested, both with and without symmetry breaking (denoted \MT, \RP, \MTSB and \RPSB, respectively), and maximum treewidths of two and five.
Treewidths were calculated with \treewidthexact~\footnote{\url{http://github.com/TCS-Meiji/treewidth-exact}} and \libtw~\cite{vanDijk2006:libtw}, and for the \MTSB and \RPSB policies, \PDG automorphisms were found with \NAUTY~\cite{McKay2104:PracGraphIso}.
As a baseline, each plan was also deordered with \EOG (Definition~\ref{def:eog}).
The flexibility of the partial plans generated by \MKTR and the \POP{}s generated by \EOG was measured by counting their instantiations (which, in the case of \EOG, is their linearisations).
The \ganak~\cite{Sharma2019:Ganak} model counter was used.

All of the planning, plan optimisation and model counting tasks above were given resource limits of $8$GB RAM and $30\mins$ at $3.2$GHz.
%
%%%%%%%%%%%%%%%%%%%%%%%%%%%%%%%%%%%%%%%%%%%%%%%%%%%%%%%%%%%%%%%%%%%%%%
\subsection{Results}\label{sec:pp-results}
%%%%%%%%%%%%%%%%%%%%%%%%%%%%%%%%%%%%%%%%%%%%%%%%%%%%%%%%%%%%%%%%%%%%%%
%
\subimport*{tables/}{model-counts-by-domain.tex}
\subimport*{tables/}{coverage-time-by-domain.tex}
%
Results are summarised in Tables~\ref{tab:pp-model-counts-domain-tw2}--\ref{tab:pp-model-counts-planner}.
Tables~\ref{tab:pp-model-counts-domain-tw2},~\ref{tab:pp-model-counts-domain-tw5} and~\ref{tab:pp-model-counts-planner} compare the flexibility of the plans found by \EOG with those found by \MKTR.
The $\#_{\EOG}$ columns indicates the average number of instantiations (i.e., linearisations) of the \POP{}s found by \EOG, and $\Delta_{\EOG}$ indicates the average percentage improvement found by each \MKTR configuration. 
As the instantiation counts can vary exponentially, averages are computed from the \emph{geometric mean}, which, for a set of positive real-valued numbers $\set{x_1,\ldots,x_n}$, is equal to $\sqrt[n]{x_1 \times \cdots \times x_n}$.
This better reflects the nature of the data, and prevents a few extremely large values skewing the central tendency of the model counts.

To ensure a meaningful comparison between \EOG and \MKTR, and between the different \MKTR configurations, $\Delta_{\EOG}$ is computed from plans for which a model count is available for every plan relaxation technique. 
Further, $\Delta_{\EOG}$ is only reported if there is a statistically significant difference between \EOG and \MKTR.\footnote{As determined by a single-tailed, paired t-test with $p < 0.05$}
For most domains and planners, such a difference could be established between \EOG and \MKTR. 
However, due to the high variance in model counts and small sample sizes, it was often impossible to establish a statistically significant difference between the different \MKTR configurations.
The cases when such a comparison is significant are marked. 
The symbols $\dagger$ and $\ddagger$ indicate a significant difference between \MT and \RP, and between a policy and its symmetry breaking variant (i.e., between \MT and \MTSB or \RP and \RPSB), respectively.

Tables~\ref{tab:pp-cov-time-domain-tw2} and~\ref{tab:pp-cov-time-domain-tw5} compare the coverage rates and execution times for \EOG and \MKTR. 
The \emph{coverage} $C$ is the proportion of plans for which \EOG found a solution.
As \MKTR defaults to \EOG (Observation~\ref{obs:pc-enc-eog}), this is also the proportion for which all \MKTR configurations found a solution.
For each configuration, the \emph{minimal coverage} $C_M$ is the proportion of plans for which a minimal $k$-treewidth CS relaxation was found. 
Finally, $T$ is the average execution time in minutes.

Domains in which no plan could be improved by either \EOG or \MKTR with any configuration have been excluded from the tables (i.e., for all \ipcdomain{pegsol}, \ipcdomain{snake}, \ipcdomain{sokoban}, \ipcdomain{termes} instances, $\#_\EOG = 1$ and all $\Delta_{\EOG} = 0\%$).
%
%%%%%%%%%%%%%%%%%%%%%%%%%%%%%%%%%%%%%%%%%%%%%%%%%%%%%%%%%%%%%%%%%%%%%%
\subsubsection{Flexibility provided by \MKTR}
%%%%%%%%%%%%%%%%%%%%%%%%%%%%%%%%%%%%%%%%%%%%%%%%%%%%%%%%%%%%%%%%%%%%%%
%
Tables~\ref{tab:pp-model-counts-domain-tw2} and~\ref{tab:pp-model-counts-domain-tw5} show that \emph{\MKTR can find significantly more flexible partial plans than \EOG}: with a maximum treewidth of two, the partial plans found by \MKTR have, on average, up to $163.2\%$ more instantiations than those found by \EOG, depending on policy.
Interestingly, there are three domains (\ipcdomain{agricola}, \ipcdomain{organic-synthesis-split} and \ipcdomain{visitall}) in which \emph{\MKTR can relax plans that \EOG cannot deorder at all}.
Of these three domains, the most striking improvement is in \ipcdomain{agricola}, where \MKTR finds partial plans with, on average, $2.4 \times 10^{20}$ instantiations while \EOG cannot improve plan flexibility at all (and indeed, neither can any of the generalised de/reordering approaches introduced in Chapter~\ref{chap:pop-maxsat}).
Other significant results are in \ipcdomain{hiking}, where the \RP policy yields a $2088.9\%$ increase in instantiations, and \ipcdomain{woodworking}, where \MTSB yields a $1050.8\%$ increase.

However, while \MKTR will, by definition, always find a plan at least as flexible as that found by \EOG (Observation~\ref{obs:pc-enc-eog}), in some domains (e.g., \ipcdomain{airport}, \ipcdomain{thoughtful}, \ipcdomain{nomystery}), it does not find a significantly more flexible one, even with a treewidth of five.

This significant increase in flexibility comes at substantial computational cost.
Tables~\ref{tab:pp-cov-time-domain-tw2} and~\ref{tab:pp-cov-time-domain-tw5} show that the average run time approaches $24\min$, and indeed of the $22560$ test cases, $67\%$ ran for the full $30\min$.
\MKTR has a run time of $<20\min$ in just four domains: \ipcdomain{ged}, \ipcdomain{cybersec}, \ipcdomain{mystery} and \ipcdomain{psr-small}.
Significantly, however, this does not result in a reduction in flexibility. 
For example, in \ipcdomain{cybersec}, \RP with treewidth five provides a $2222.9\%$ increase over \EOG in $3.05\min$, and in \ipcdomain{mystery}, \MT with treewidth two provides a $56.6\%$ increase over \EOG in $14.11\min$.
%
%%%%%%%%%%%%%%%%%%%%%%%%%%%%%%%%%%%%%%%%%%%%%%%%%%%%%%%%%%%%%%%%%%%%%%
\subsubsection{Effect of Relaxation Policy}
%%%%%%%%%%%%%%%%%%%%%%%%%%%%%%%%%%%%%%%%%%%%%%%%%%%%%%%%%%%%%%%%%%%%%%
%
While Tables~\ref{tab:pp-model-counts-domain-tw2} and~\ref{tab:pp-model-counts-domain-tw5} suggest that overall, \RP is more successful than \MT, this difference is not statistically significant.
There are, however, significant differences in twelve domains (marked with $\ddagger$), and these results suggest that some policies are suited to certain domains.
For example, in \ipcdomain{hiking}, \RP is the most successful. 
With a treewidth of two and five, $\Delta_\EOG$ is $2033.5\pp$\footnote{A percentage point $\pp$ measures the difference between two percentages, e.g., $50\%$ and $55\%$ differ by $10\%$ but $5\pp$.} and $2345.2\pp$ higher than \MT's, respectively. 
In contrast, \MT is the most effective in \ipcdomain{woodworking}, where, with a treewidth of two and five, its $\Delta_\EOG$ is $506.6\pp$ and $1094\pp$ greater than \RP's, respectively.

Further, in \ipcdomain{visitall}, \MT with a treewidth of two finds $14.3\%$ more instantiations than \EOG while \RP offers no improvement, and in \ipcdomain{gripper} the opposite is true: \RP with a treewidth of five finds $37.6\%$ more instantiations than \EOG, while \MT provides no improvement.
%
%%%%%%%%%%%%%%%%%%%%%%%%%%%%%%%%%%%%%%%%%%%%%%%%%%%%%%%%%%%%%%%%%%%%%%
\subsubsection{Effect of Symmetry Breaking}
%%%%%%%%%%%%%%%%%%%%%%%%%%%%%%%%%%%%%%%%%%%%%%%%%%%%%%%%%%%%%%%%%%%%%%
%
Comparing \MT and \RP with \MTSB and \RPSB suggests that symmetry breaking results in little overall difference in plan flexibility or run time.
Furthermore, in the domains for which symmetry breaking does influence results, its effects are not always beneficial, suggesting that the search space reduction it provides is sometimes not enough to counteract the additional overhead of detecting symmetries.

Tables~\ref{tab:pp-model-counts-domain-tw2} and~\ref{tab:pp-model-counts-domain-tw5} show that, when considered over all test cases, symmetry breaking makes no statistically significant difference to the flexibility of the final partial plans.
The only domains in which it does produce a significant change (marked by $\dagger$) are \ipcdomain{cybersec} where, with a treewidth of five, \RPSB's $\Delta_\EOG$ is $682.2\pp$ \emph{lower} than \RP's, and \ipcdomain{ged}, where, also with a treewidth of five, symmetry breaking results in a $2.6\pp$ increase in $\Delta_\EOG$ for \MT.
The only planner for which symmetry breaking significantly effects results is \lama, where \RPSB with a treewidth of five improves on \EOG by $395.2\%$ to \RP's $408.3\%$.

Tables~\ref{tab:pp-cov-time-domain-tw2} and~\ref{tab:pp-cov-time-domain-tw5} show that symmetry breaking has no significant effect on the proportion of plans that \MKTR solves optimally ($C_M$), and while in some domains it has an effect on the average run time (marked by $\dagger$), in all cases the difference is under ten seconds.

A possible reason for this ineffectiveness is that \MKTR stops on finding a local minimum, rather than backtracking and attempting to find a better solution.
As a result, symmetry breaking can only prune single states, and not entire branches.
%
%%%%%%%%%%%%%%%%%%%%%%%%%%%%%%%%%%%%%%%%%%%%%%%%%%%%%%%%%%%%%%%%%%%%%%
\subsubsection{Effect of Treewidth}
%%%%%%%%%%%%%%%%%%%%%%%%%%%%%%%%%%%%%%%%%%%%%%%%%%%%%%%%%%%%%%%%%%%%%%
%
Tables~\ref{tab:pp-model-counts-domain-tw2} and~\ref{tab:pp-model-counts-domain-tw5} demonstrate that allowing \MKTR to search for high-treewidth constraints can yield more flexible partial plans with more instantiations.
Overall, increasing the maximum treewidth to five yields an increase in $\Delta_\EOG$ of $110.1\pp$ and $139.1\pp$ for \MT and \RP, respectively.

Some domains show more extreme differences.
In \ipcdomain{cybersec}, increasing the treewidth results in a $1579.7\pp$ and $2142.8\pp$ improvement in $\Delta_\EOG$ for \MT and \RP, respectively, and in \ipcdomain{tpp}, neither \MT nor \RP provide a statistically significant benefit with a treewidth of two, but with five they improve on \EOG by $110.9\%$ and $288.2\%$, respectively.

While there is no domain in which increasing the treewidth to five resulted in an average decrease in instantiations, there are $166$ individual plans for which using the same policy but increasing the treewidth to five resulted in a \emph{less flexible} plan.
There are two possible reasons for this: either the increased cost of treewidth calculations (from $n^2$ to $n^5$) slow \MKTR, or the additional branches made available by increasing the treewidth in fact lead to worse solutions from which \MKTR cannot recover as it always stops upon finding a local optimum.

Interestingly, the flexibility benefit of increasing the treewidth comes at little instantiation cost.
With a treewidth of two, $82.8\%$ of the ($1200$ randomly selected) final partial plans can be instantiated in under $250$ms using the \maple~\cite{Liang16:MapleSAT} \SAT solver, and with a treewidth of five this reduces to $78.3\%$ (although in both cases some outliers timed out after half an hour).
%
%%%%%%%%%%%%%%%%%%%%%%%%%%%%%%%%%%%%%%%%%%%%%%%%%%%%%%%%%%%%%%%%%%%%%%
\subsubsection{Effect of Planner}
%%%%%%%%%%%%%%%%%%%%%%%%%%%%%%%%%%%%%%%%%%%%%%%%%%%%%%%%%%%%%%%%%%%%%%
%
\subimport*{tables/}{model-counts-by-planner.tex}
%
Table~\ref{tab:pp-model-counts-planner} shows that the source of the input plan influences the final flexibility.
Plans generated by \lama benefit the most from \MKTR. 
With a treewidth of two, \MKTR with the \MT and \RP policies produces $211.6\%$ and $261\%$ more instantiations than \EOG, and with a treewidth of five, this increases to $317.8\%$ and $408.3\%$, respectively.

As discussed in Section~\ref{sec:reins-results}, \madagascar produces \emph{parallel plans} (Definition~\ref{def:parallel-plan}), that is, generalised plans of the form $T = \tup{S_1,\ldots,S_n}$ where each step $S_i$ is a set of actions that can be executed in any order (or indeed, simultaneously), but must precede all actions in any subsequent steps.
On average, \EOG increases the number of linearisations of the parallel plans produced by \madagascar from $1324.78$ to $57628.6$. 
With a treewidth of two, \MKTR with the \MT and \RP policies further improves on this by $79.9\%$ and $50.7\%$, respectively, and a treewidth of five increases this to $195\%$ and $179.4\%$, respectively.
Thus, while \lama plans benefit the most from \MKTR, optimised \madagascar plans are the most flexible.
