\section{Introduction}\label{sec:reins-intro}

The least commitment approach to planning~\cite{Weld94:LeastCommitment} aims to provide a scheduler or executing agent with additional flexibility at execution time. 
One way to achieve this is to post-process a ground, totally ordered plan into a minimally constrained partial-order plan (\POP). 
This approach, known as \emph{plan deordering} or \emph{plan reordering} depending on the type of post-processing applied, has been extensively studied from both a technical~\cite{Backstrom-CompAspects,Aghighi2015:TractablePlanningPolytree} and practical~\cite{Kambhampati:2004:ExplBasedGeneralisation,Muise2016-PopMaxSAT,SiddiquiHaslum2012-Block,Say2016:MathematicalPOP} perspective (Section~\ref{sec:pop-opt-related-work}).

An aspect of least commitment which has received less attention is the optimisation of an agent's commitment to the domain objects that are used in the course of executing a plan.
By their nature, deordering and reordering processes can only remove or modify the agent's commitment to the \emph{timing} of actions: the resulting optimised plan will contain the same actions, and therefore make use of the same domain objects, as the original.
This over-commitment to a particular set of objects can have a number of detrimental effects on the agent's execution-time options, including an over-commitment to a particular action ordering.

As discussed in Section~\ref{sec:planning}, an action can be defined as a combination of an operator (a schematised action type) and a set of bindings for the operator's variables, that is, an action is an \emph{instantiation} of an operator.
A \emph{reinstantiation} of an action is therefore a modification of its variable bindings, that is, a modification of the domain objects used by the agent in the course of executing the action.
This chapter will introduce the notions of \emph{reinstantiated deorderings} and \emph{reinstantiated reorderings}: transformations of a plan under which ordering constraints can be removed or arbitrarily modified, respectively, \emph{and the plan's actions' variable bindings can be changed}.
The motivation behind these transformations is the possibility of finding a new set of variable bindings that allows for greater minimisation of the plan's ordering constraints--resulting in less commitment and more flexibility--than would be possible had the bindings remained unchanged.

Allowing a plan optimisation process to change variable bindings results in an exponentially larger search space. 
However, this space is often highly \emph{symmetrical}, meaning that a plan has many transformations that are identical, and equally optimal, up to an irrelevant permutation of, for example, operators of the same type, or interchangeable domain objects.
As an optimisation algorithm need only visit one element from each set of symmetrical plans, search times can be greatly reduced through the addition of symmetry breaking constraints.
Thus, this chapter will introduce a number of symmetry breaking techniques to aid in the reinstantiation task.
In particular, the notion of a \emph{causal structure symmetry} is introduced, a symmetry that results from interchangeable (potential) causal links between operator preconditions and postconditions.

\subsection{Example}\label{sec:pop-reins-example}

\subimport*{graphics/}{rovers-plans-table.tex}

Consider a small planning task from (a reduced version of) the \IPC \ipcdomain{rovers} domain, in which a fleet of rovers must navigate the surface of a planet, collecting soil and rock samples.
% and take photos, and then transmit the data back to Earth.
The domain objects comprise two rovers ($\rover_1$ and $\rover_2$) and three connected waypoints ($\wayp_1$--$\wayp_3$).
Both rovers begin at $\wayp_1$, there are soil and rock samples at $\wayp_2$ and $\wayp_3$, respectively, and paths connect all waypoints.
The goal is to gather both samples.

Classical plan $P_1$ in Figure~\ref{fig:rovers-p1}\footnote{Actions representing the initial state and goal have been omitted for space, and subscripts have been used to better distinguish between actions of the same type.} is an optimal solution to this problem: $\rover_1$ navigates to $\wayp_2$ and collects the soil sample, then navigates to $\wayp_3$ and collects the rock sample.
%The causal structure of $P_1$ can be described in the context of the \emph{producer-consumer-threat} formalism (\PCT, Section~\ref{sec:pct}).
An examination of the causal structure of $P_1$ shows that its ordering constraints cannot be modified without invalidating the plan.
The preconditions of both $\samplesoil$ and $\navigate_2$ require that $\rover_1$ be at $\wayp_2$, a condition which is brought about only by $\navigate_1$. 
Thus, for $P_1$ to remain valid, $\navigate_1$ must precede both $\samplesoil$ and $\navigate_2$, and for similar reasons $\navigate_2$ must precede $\samplerock$. 
Additionally, $\navigate_2$ moves $\rover_1$ away from $\wayp_2$; it therefore \emph{threatens} the causal link between $\navigate_1$ and $\samplesoil$ and cannot be ordered between them.
%As $\navigate_2$ cannot precede $\navigate_1$, it follows that $\samplesoil$ must precede $\navigate_2$.
Thus, validity-preserving ordering constraints exist between all actions in $P_1$.
As a result, standard de/reordering techniques cannot provide any additional flexibility, as any relaxation or modification of $P_1$'s ordering will render the plan invalid.

However, if $P_1$'s variable bindings are allowed to change, then a more relaxed ordering can be found.
Classical plan $P_2$ in Figure~\ref{fig:rovers-p2} is a modification of $P_1$ that, while still using the same operator \emph{types}, instead uses rover $\rover_2$ to collect the rock sample.
%%
This modification allows its ordering constraints to be relaxed: the actions can be executed in any order so long as $\navigate_1$ precedes $\samplesoil$, and $\navigate_2$ precedes $\samplerock$. 
The \POP in Figure~\ref{fig:rovers-p3} is such a relaxation.

\bigskip

The rest of this chapter will continue as follows.
Section~\ref{sec:reins-pop-opt} defines new optimality criteria for \POP{}s which take all possible sets of variable bindings into consideration.
Section~\ref{sec:reins-encodings} introduces a technique for computing optimally reinstantiated \POP{}s by encoding the problem into a \MAXSAT instance, and Section~\ref{sec:reins-sb} introduces two symmetry-breaking techniques to improve this encoding.
Section~\ref{sec:reins-eval} experimentally assesses the benefits provided by reinstantiated reordering through a comparison with the de/reordering encodings of \citet{Muise2016-PopMaxSAT} and the non-optimal but effective \EOG technique of \citet{Kambhampati:2004:ExplBasedGeneralisation} (Section~\ref{sec:pop-opt-related-work}).

Results show that optimising both bindings and orderings incurs a much greater computational cost than optimising orderings alone. 
However, when (even non-optimal) reinstantiated de/reorderings are found, they are significantly more flexible than plans optimised with standard methods.
