\section{Symmetry Breaking}\label{sec:reins-sb}
%
A search or optimisation problem displays \emph{symmetry} when subsets of its solutions are identical up to an irrelevant permutation of variables and/or constants.
As a sound and complete search algorithm need only explore one element per set, exploiting symmetries can exponentially reduce the search times (Section~\ref{sec:symmetries}).
Here, symmetries are defined \emph{syntactically}, that is, as automorphisms (Section~\ref{sec:automorphisms}) of a problem description that, when applied to a solution, preserves its validity and optimality, and are broken \emph{statically}, that is, by extending the problem description with constraints that rule out redundant areas of the search space.

Reinstantiated de/reordering differs from standard de/reordering by allowing variable bindings to change.
This raises the possibility of finding more flexible relaxations of a \POP, but also creates an exponentially larger search space.
Thus, any technique for reducing search times is welcome.

This section introduces extensions to \MR and \MRR that exploit three types of symmetry in \POP{}s: \MRRO breaks \emph{operator symmetries}, which occur when a \POP contains multiple operators of the same type, \MRO breaks \emph{action symmetries}, which occur when those operators also have the same parameters, and \MRRC breaks \emph{causal structure symmetries}, which are produced by equivalent combinations of (potential) causal links.
%
%%%%%%%%%%%%%%%%%%%%%%%%%%%%%%%%%%%%%%%%%%%%%%%%%%%%%%%%%%%%%%%%%%%%%%%%%%%%%%%%
\subsection{Breaking Operator and Action Symmetries}\label{sec:break-op-symms}
%%%%%%%%%%%%%%%%%%%%%%%%%%%%%%%%%%%%%%%%%%%%%%%%%%%%%%%%%%%%%%%%%%%%%%%%%%%%%%%%
%
The \MRRO encoding derives from the observation that swapping the ordering and binding constraints of operators of the same type will not affect the \POP{}'s validity or optimality.
For example, consider two simple symmetrical plans from a path-finding domain, constructed from two instances of the $\move$ operator: $\tup{\move_1(1, 2), \move_2(2, 3)}$ and $\tup{\move_2(1, 2), \move_1(2, 3)}$.\footnote{Subscripts are used to better distinguish between different instances of actions or operators of the same type.}
The plans are identical up to swapping the orderings and bindings of $\move_1$ and $\move_2$.

Despite permuting both orderings and bindings, operator symmetries can, in many cases, be broken with ordering constraints alone.
For example, the symmetry above can be broken by simply requiring that $\move_2 \not\prec \move_1$.
The \MRRO encoding generalises this approach by selecting a linearisation of operators $\tup{\optr_1,\ldots,\optr_n}$ and requiring that for all $1 \leq i < j \leq n$ \ST $\name(\optr_i) = \name(\optr_j)$, $\optr_j \not\prec \optr_i$.

This constraint is \emph{correct} (i.e., at least one element is left per symmetry class) as every \POP in which $\optr_j \prec \optr_i$ has a symmetric alternative in which the two operators' orderings and bindings have been swapped. 
However, it is not \emph{complete} as it cannot rule out symmetrical bindings of unordered operators.
For example, plans $P_4$ and $P_5$ in Figure~\ref{fig:synth-example} are identical up to a permutation of $A(x_1)$ and $A(x_2)$, but requiring that $A(x_1) \not\prec A(x_2)$ will rule out neither.

As discussed in Section~\ref{sec:mrd-mrr-enc}, rather than breaking symmetries by adding negated unit clauses (e.g., $\neg \precp{\move_2}{\move_1}$) it is more efficient to remove those propositions from the encoding entirely, and update the ``allowable  orderings'' $\deoprec$.
Thus, under \MRRO, $\deoprec$ does not contain any orderings between operators of the same type that contradict an arbitrarily chosen linearisation.
Unlike standard approaches which break symmetries with additional constraints, this approach results in a \emph{smaller} encoding:
%
\begin{defn}\label{def:mrro}
If $P = \tup{\opset, \sub, \precrel}$ is a \POP and $\prec'$ is an arbitrary linearisation of $P$, then \MRRO encodes $P$ into a partially-weighted \MAXSAT instance through Formulae~\ref{eq:mrr-asymm}--\ref{eq:mrr-soft-order}, with the allowable orderings $\deoprec = \opset^2 \setminus \set{(\optr_2, \optr_1) : \name(\optr_1) = \name(\optr_2), \optr_1 \prec' \optr_2})$.
\end{defn}
%
\citet{Say2016:MathematicalPOP}'s work on plan de/reordering (Section~\ref{sec:milp-reorder}) addresses a similar form of symmetry that occurs when a plan contains multiple identical \emph{actions}.
Operator and action symmetries are subtly different: the former occur when plan steps have the same type, the latter when they also have the same bindings.
Nevertheless, the \MRO encoding breaks action symmetries in a similar way. 
If $\prec'$ is an arbitrary linearisation of the \POP, then Formula~\ref{eq:mra-symm-break} defines hard clauses that break action symmetries:
%
\begin{align}
& \bigwedge_{\clapstack{\actn_1(\vec{t}), \actn_2(\vec{u}): \\ \name(\actn_1) = \name(\actn_2), \\ \vec{t} = \vec{u}, \actn_1(\vec{t}) \prec' \actn_2(\vec{u})}} \neg \precp{\actn_2}{\actn_1}. \label{eq:mra-symm-break}
\end{align}%
%
The \MRO encoding extends \MR by breaking action symmetries:
%
\begin{defn}\label{def:mra} \MRO encodes a \POP into a partial weighted \MAXSAT instance through Formulae~\ref{eq:mr-irr}--\ref{eq:mr-soft-order} and~\ref{eq:mra-symm-break}.
\end{defn}
%
%%%%%%%%%%%%%%%%%%%%%%%%%%%%%%%%%%%%%%%%%%%%%%%%%%%%%%%%%%%%%%%%%%%%%%%%%%%%%%%%
\subsection{Breaking Causal Structure Symmetries}\label{sec:causal-struct-symms}
%%%%%%%%%%%%%%%%%%%%%%%%%%%%%%%%%%%%%%%%%%%%%%%%%%%%%%%%%%%%%%%%%%%%%%%%%%%%%%%%
%
A \POP{}'s \emph{causal structure} is an implicit set of \emph{unthreatened causal links} (Definition~\ref{def:pop-clinks}) that associate operator postconditions (i.e., producers) with causally dependent preconditions (i.e., consumers). 
A causal structure can be represented as a matrix (Section~\ref{sec:notation}) of Boolean variables, $\mat{C}$, that associates producers and consumers with rows and columns, respectively, with $\mat{C}[i][j] = 1$ indicating that there is an unthreatened causal link between producer $i$ and consumer $j$.
%Under \POCL validity, every consumer must be causally linked with (i.e., preceded by, a bound to the same value as) a producer, and this link must be unthreatened (i.e., )
%A causal link exists between a precondition $q(\vec{t})$ of operator $\optr_c$ and postcondition $q(\vec{u})$ of operator $\optr_p$ \IFF $\optr_p$ is ordered before $\optr_c$ and $\vec{t}$ and $\vec{u}$ are bound to the same objects; the link is unthreatened \IFF any operator with $\neg q(\vec{s})$ as a postcondition is either not ordered between $\optr_p$ and $\optr_c$, or not bound to the same domain objects.

A \emph{causal structure symmetry} occurs when a plan's operators contain combinations of pre/postconditions that are \emph{functionally equivalent} in that their causal links can be swapped without affecting the plan's validity or optimality.
Such symmetries can be expressed as permutations of rows and/or columns of variables in $\mat{C}$, and efficiently broken with lexicographic constraints over the same.

As discussed above, \MRRO breaks operator symmetries, which occur when a \POP contains multiple operators of the same type.
However, a common form of symmetry not removed by \MRRO is object symmetry, which occurs when a \POP's variables' domains contain (combinations of) interchangeable values.
The \MRRC encoding is motivated by the observation that \emph{breaking causal structure symmetries simultaneously breaks both operator and object symmetries}.

For example, Figure~\ref{fig:synth-example} depicts two symmetrical plans from a synthetic domain, with causal structures represented by both arrows and matrices.
The two instances of $A$ create an operator symmetry, meaning that swapping the bindings and orderings of $A(x_1)$ and $A(x_2)$ will produce an equally valid and optimal plan.
This can be also described as a causal structure symmetry: modifying the plans so that any producer supporting $p(x_1)$ instead supports $p(x_1)$, and any consumer supported by $q(x_1)$ is instead supported by $q(x_2)$, and \emph{vice versa}, will not affect their validity or optimality.
Plan $P_5$ is the result of applying the above symmetry to $P_4$.

The plans also display an object symmetry that is expressible as a causal structure symmetry: as objects $1$ and $2$ are interchangeable, modifying the plans so that all consumers supported by $p(1)$ are instead supported by $p(2)$, and \emph{vice versa}, preserves validity and optimality.

A key element of the causal structure symmetries described above is that they \emph{swap multiple producers and consumers simultaneously}, and are thus \emph{multi-row-column symmetries} (Definition~\ref{def:multi-row-col-symm}) over $\mat{C}$. 
As such symmetries \emph{cannot} in general be broken with standard approaches such as \dblex~\cite{Flener2002:DoubleLex} or \snakelex~\cite{Grayland2009:SnakeLex} (Section~\ref{sec:symm-break-matrix}), the \multilex constraint introduced in Chapter~\ref{chap:symmetry-breaking} (Definition~\ref{def:multilex}) must be used instead.

For the object symmetry, this resolves to a simple requirement that the rows in $\mat{C}$ associated with $p(1)$ and $p(2)$ are lexicographically ordered, which holds for $P_5$ but not $P_4$.
As the operator symmetry permutes producers and consumers simultaneously, the resulting constraint is more complex, resolving to a lexicographic comparison of the shaded areas in Figure~\ref{fig:synth-example}.
As the shaded area in $P_5$ lexicographically precedes that in $P_4$, the constraint once again holds for $P_5$ but not $P_4$.

Causal structure symmetries can be detected using the standard technique introduced by \citet{Crawford1996:SymmetryPredicates} (Section~\ref{sec:symm-det-break}), in which a problem's symmetries are found by computing the automorphisms of a structurally equivalent coloured graph.
This section defines symmetry \WRT \POP{}s, introduces the \emph{plan description graph}, and then defines \MRRC, an extension to \MRR that breaks causal structure symmetries with \multilex.

%%%%%%%%%%%%%%%%%%%%%%%%%%%%%%%%%%%%%%%%%%%%%%%%%%%%%%%%%%%%%%%%%%%%%%%%%%%%%%%%
\subsubsection{Symmetrical Partial-Order Plans}
%%%%%%%%%%%%%%%%%%%%%%%%%%%%%%%%%%%%%%%%%%%%%%%%%%%%%%%%%%%%%%%%%%%%%%%%%%%%%%%%
%
A symmetry is an automorphism of a problem description that preserves the problem's solutions.
%When finding minimal reinstantiated reorderings, the input to the problem is a \POP, and the (candidate) solutions are \POP{}s with modified bindings and ordering constraints.
As the definition of a minimum reinstantiated reordering (Definition~\ref{def:opt-reins-re-deorder}) makes no reference to the variable bindings or ordering constraints of the input \POP, in this case the problem description is simply the input \POP{}'s operators, $\opset$.
A symmetry of a \POP is therefore a permutation of its variables and constants that leaves $\opset$ unchanged and preserves the validity and optimality of any reinstantiated reorderings:\footnote{The notation $\perm(\eta)$ is used to denote the result of simultaneously replacing every variable or constant $t$ in $\eta$ with $\perm(t)$, while preserving the structure (or type) of $\eta$.}

\begin{defn}\label{def:pop-symm} Permutation $\perm$ is a \defterm{symmetry} of \POP $P\!=\!\tup{\opset, \sub, \precrel}$ \IFF:
\begin{itemize}
	\item $\domain(\perm) = \vars(\opset) \cup \consts(\opset)$,
	\item $\perm(\opset) = \opset$, and
	\item if $Q = \tup{\opset, \sub', \precrel'}$ is a reinstantiated reorder of $P$, then $Q$ is valid \IFF $\perm(Q)$ is valid and $\card{\precrel'} = \card{\perm(\precrel')}$.
\end{itemize}
\end{defn}
%
A \POP{}'s symmetries form a \emph{symmetry group} (Section~\ref{sec:permutations}) that partitions the space of reinstantiated reorderings into sets of symmetrically equivalent solutions.
A \POP $P$ with symmetry group $\group$ is symmetrical to a \POP $Q$ \IFF there is some $\perm \in \group$ \ST $\perm(P) = Q$.
%
%%%%%%%%%%%%%%%%%%%%%%%%%%%%%%%%%%%%%%%%%%%%%%%%%%%%%%%%%%%%%%%%%%%%%%%%%%%%%%%%
\subsubsection{Detecting \POP Symmetries}\label{sec:symm-breaking-pdg}
%%%%%%%%%%%%%%%%%%%%%%%%%%%%%%%%%%%%%%%%%%%%%%%%%%%%%%%%%%%%%%%%%%%%%%%%%%%%%%%%
%
A \POP{}'s symmetries are detected by finding the automorphisms of its \emph{plan description graph} (\PDG), an undirected coloured graph that encodes the relevant aspects of its structure.
This approach derives from the commonly used \emph{problem description graph} (Section~\ref{sec:struct-symm-planning}), which is used to find symmetries in planning instances in the context of heuristic state-space planning. 
The \PDG is an adaptation of this for \POP optimisation problems:

\begin{defn}\label{def:pdg} The \defterm{plan description graph} of a \POP $P = \tup{\opset, \sub, \precrel}$ is an undirected coloured graph $\pdg_P = \tup{\vertset, \edgeset, \colfunc}$ with vertices $\vertset$, edges $\edgeset$ and colour function $\colfunc$ defined as follows:
\begin{itemize}
	\item $\vertset$ is the smallest set \ST:
	\begin{itemize}
		\item if $t \in \vars(\opset) \cup \consts(\opset)$ then $\vvert{t} \in \vertset$,
		\item if $\optr \in \opset$ then $\opvertpre{\optr}, \opvertpost{\optr} \in \vertset$, and
		\item if $\optr \in \opset$, $q(\vec{t}) \in \post(\optr) \cup \pre(\optr)$ and $1 \leq n \leq \card{\vec{t}}$ then $\litvert{q(\vec{t})}{n} \in \vertset$.
	\end{itemize}
	\pagebreak
	\item $\edgeset$ is the smallest relation \ST:
	\begin{itemize}
		\item if $\optr(\vec{x}) \in \opset$ then $\gedge{\opvertpre{\optr(\vec{x})}, \vvert{\vec{x}[1]}}, \gedge{\opvertpost{\optr(\vec{x})}, \vvert{\vec{x}[1]}} \in \edgeset$,
		\item if $\optr(\vec{x}) \in \opset$ and $1 \leq i < \card{\vec{x}}$ then $\gedge{\vvert{\vec{x}[i]}, \vvert{\vec{x}[i+1]}} \in \edgeset$,
		\item if $\optr \in \opset$ and $q(\vec{t}) \in \pre(\optr)$ then $\gedge{\opvertpre{\optr}, \litvert{q(\vec{t})}{1} } \in \edgeset$,
		\item if $\optr \in \opset$ and $q(\vec{t}) \in \post(\optr)$, $\gedge{\opvertpost{\optr}, \litvert{q(\vec{t})}{1} } \in \edgeset$, and
		\item if $\optr \in \opset$, $q(\vec{t}) \in \post(\optr) \, \cup \, \pre(\optr)$ and $1 \leq i < \card{\vec{t}} \\$ then $\gedge{\litvert{q(\vec{t})}{i}, \litvert{q(\vec{t})}{i+1}} \in \edgeset$.
	\end{itemize}
	\item For all $\gvert_1,\gvert_2 \in \vertset$, $\colfunc(\gvert_1) = \colfunc(\gvert_2)$ \IFF:
	\begin{itemize}
		\item there is a $t_1, t_2 \in \vars(\opset) \cup \consts(\opset)$ \ST $\type(t_1) = \type(t_2)$, $\gvert_1 = \vvert{t_1}$ and $\gvert_2 = \vvert{t_2}$,
		\item there is an $\optr_1, \optr_2 \in \opset$ \ST $\name(\optr_1) = \name(\optr_2)$ and either $\gvert_1 = \opvertpre{\optr_1}$ and $\gvert_2 = \opvertpre{\optr_2}$, or $\gvert_1 = \opvertpost{\optr_1}$ and $\gvert_2 = \opvertpost{\optr_2}$, or
		\item there is a $q(\vec{t}_1), q(\vec{t}_2), i, j$ \ST $\gvert_1 = \litvert{q(\vec{t}_1)}{i}$ and $\gvert_2 = \litvert{q(\vec{t}_2)}{j}$.
	\end{itemize}
\end{itemize}
\end{defn}
%
% vertices and colours
A vertex $\vvert{t}$ is introduced for each variable and constant, a pair of vertices $\opvertpre{\optr}$ and $\opvertpost{\optr}$ are introduced for each operator, representing its sets of preconditions and postconditions, respectively, and for every pre/postcondition of every operator the vertices $\litvert{q(\vec{t})}{1},\ldots,\litvert{q(\vec{t})}{n}$ are introduced, where $n$ is the arity of $q$. 
A different colour is introduced for every type of variable, constant and predicate, and for each operator type two colours are introduced, for its pre/postconditions, respectively.
All constant, variable and operator vertices are coloured by type, and all pre/postcondition vertices are coloured according to the type and polarity of the predicate symbol.

% edges 
Each operator's parameters are represented by a chain of vertices, with the first such parameter connected to the vertices representing the operator pre/postconditions.
For example, if $\optr(\vec{x}) \in \opset$ and $\vec{x} = \tup{x_1,\ldots,x_n}$, then $\gedge{\opvertpre{\optr(\vec{x})}, \vvert{x_1}}$, $\gedge{\opvertpost{\optr(\vec{x})}, \vvert{x_1}}$, $\gedge{\vvert{x_1}, \vvert{x_2}}, \ldots, \gedge{\vvert{x_{n-1}}, \vvert{x_n}} \in \edgeset$.
The vertices representing the pre/postconditions are also connected in a chain, with the first such vertex connected to that representing either the operator pre/postconditions as appropriate.
For example, if $q(\vec{t}) \in \post(\optr(\vec{x}))$ and $\vec{t} = \tup{t_1,\ldots,t_n}$, then $\gedge{\opvertpost{\optr(\vec{x})}, \litvert{q(\vec{t})}{1}}, \gedge{\litvert{q(\vec{t})}{1}, \litvert{q(\vec{t})}{2}}, \ldots, \gedge{ \litvert{q(\vec{t})}{n-1}, \litvert{q(\vec{t})}{n} } \in \edgeset$.
Additionally, to represent the specific terms found in each pre/postcondition, each vertex is connected to the term found in that position, that is, $\gedge{ \litvert{q(\vec{t})}{1}, \cvert{t_1} }$, $\ldots,$ $\gedge{ \litvert{q(\vec{t})}{n}, \cvert{t_n} } \in \edgeset$.

\subimport*{graphics/}{pdg-example.tex}
%
For example, Figure~\ref{fig:pdg-example} depicts a \POP and its \PDG.
The component on the left of the \PDG represents the initial state, goal and the constants ($1$,$2$ and $3$), and that on the right represents the operator $A(x,y)$. 
Vertex borders indicate colours.

The automorphisms of a \POP's \PDG correspond to symmetries of the \POP.
If $\perm \in \auts(\pdg_P)$, then $\tperm$ represents its translation into a permutation over the terms (i.e., variables and constants) appearing in $P$:
%
\begin{defn}\label{def:aut-trans} If $P = \tup{\opset, \sub, \precrel}$ is a \POP and $\perm \in \auts(\pdg_P)$, then $\tperm$ is the permutation \ST $\tperm(t_1) = t_2$ \IFF $\perm(\vvert{t_1}) = \vvert{t_2}$ and $t_1, t_2 \in \vars(\opset) \cup \consts(\opset)$.
\end{defn}
%
For example, in Figure~\ref{fig:pdg-example}, the objects $2$ and $3$ are interchangeable. 
From Figure~\ref{fig:pdg-example-graph}, $P_6$'s \PDG has a single automorphism, $\perm$, that swaps $\cvert{2}$, $\litvert{p(1,2)}{1}$ and $\litvert{p(1,2)}{2}$ with $\cvert{3}$, $\litvert{p(1,3)}{1}$ and $\litvert{p(1,3)}{2}$, respectively.
This translates into the \POP symmetry $\tperm = \cycle{2, 3}$, as expected.

The theorem below demonstrates the correctness of this approach:
%
\begin{restatable}{theorem}{pdgthrm}\label{thrm:pdg}
\csprob{Plan Description Graph}
If $P$ is a \POP and $\perm \in \auts(\pdg_P)$ then $\tperm$ is a symmetry of $P$.
\end{restatable}	
%
Of particular interest here are symmetries permute the \POP{}'s causal structure.
A \emph{consumer symmetry} or \emph{producer symmetry} swaps the terms appearing in a pair of operator pre/postconditions, respectively, and a \emph{causal structure symmetry} is a generalisation that swaps multiple producers and consumers simultaneously:
%
\begin{defn}\label{def:cs-symm} Let $P = \tup{\opset, \sub, \precrel}$ be a \POP with symmetry $\tperm$, and $\tup{\vec{t}_1,\ldots,\vec{t}_n}$ and $\tup{\vec{u}_1,\ldots,\vec{u}_m}$ be the pre/postcondition parameters for all operators in $\opset$.
Then:
\begin{itemize}
	\item $\tperm$ is \defterm{producer symmetry} \IFF there exists an $i,j$ \ST $\tperm$ swaps $\vec{u}_i$ and $\vec{u}_j$ and fixes all other terms,
	\item $\tperm$ is \defterm{consumer symmetry} \IFF there exists an $i,j$ \ST $\tperm$ swaps $\vec{t}_i$ and $\vec{t}_j$ and fixes all other terms, and
	\item $\tperm$ is \defterm{causal structure symmetry} \IFF it is expressible as $\tperm = \tperm_1 \permcomp \cdots \permcomp \tperm_k$ where each $\tperm_i$ is either a producer or consumer symmetry.
\end{itemize}
\end{defn}
%
For example, in Figure~\ref{fig:pdg-example}, $\tup{1,2}$ and $\tup{1,3}$ are the parameters for $\initoptr$'s postconditions.
As $\tperm = \cycle{2, 3}$ is a symmetry of $P_6$, and $\tperm(\tup{1,2}) = \tup{1,3}$ and \emph{vice versa}, $\tperm$ is a causal structure symmetry. 
%
%%%%%%%%%%%%%%%%%%%%%%%%%%%%%%%%%%%%%%%%%%%%%%%%%%%%%%%%%%%%%%%%%%%%%%%%%%%%%%%%
\subsubsection{Breaking Causal Structure Symmetries with \multilex}\label{sec:symm-breaking-ml}
%%%%%%%%%%%%%%%%%%%%%%%%%%%%%%%%%%%%%%%%%%%%%%%%%%%%%%%%%%%%%%%%%%%%%%%%%%%%%%%%
%
Causal structure symmetries can be broken by applying \multilex (as described in Section~\ref{sec:multi-lex} of Chapter~\ref{chap:symmetry-breaking}) to a matrix representation of a causal structure.
In the \MRR and \MRD encodings above, the optimised \POP's causal structure is determined by the values of the variables in $\pclpset$ (Definition~\ref{def:maxsat-props-sets}), with each $\pclps{\vec{t}}{\vec{u}} \in \pclpset$ indicating that there is some $\optr_p, \optr_c$ and $q$ \ST causal link $\tup{\optr_p, q(\vec{t}), \optr_c, q(\vec{u})}$ is unthreatened in the output optimised \POP.
These variables can be arranged into a matrix, $\mat{C}$, that associates producers and consumers with rows and columns, respectively, with $\mat{C}[i][j] = 1$ \IFF producer $i$ supports consumer $j$:
%
\begin{defn}\label{def:pop-matrices} Let $P = \tup{\opset, \sub, \precrel}$ be a \POP, and $\tup{\vec{t}_1,\ldots,\vec{t}_n}$ and $\tup{\vec{u}_1,\ldots,\vec{u}_m}$ be the pre/postcondition parameters for all operators in $\opset$.
Then, $\mat{C}$ is an $n \times m$ matrix of variables from $\pclpset$ \ST $\mat{C}[i][j]$ contains variable $\pclps{\vec{t}_i}{\vec{u}_j}$.
\end{defn}
%
Causal structure symmetries can be expressed as permutations of $\mat{C}$'s rows and/or columns.
If $\tperm$ is a causal structure symmetry, then $\cperm$ denotes its transformation into a multi-row-column symmetry over $\mat{C}$:\footnote{From Definition~\ref{def:row-col-perm}, $\rowperm{i}{j}$ and $\colperm{i}{j}$ denote permutations that exchange the variables in rows $i$ and $j$, or columns $i$ and $j$, respectively, and fix all others.}
%
\begin{defn}\label{def:mrc-cs-symm} Let $P = \tup{\opset, \sub, \precrel}$ be a \POP with symmetry $\tperm$, and $\tup{\vec{t}_1,\ldots,\vec{t}_n}$ and $\tup{\vec{u}_1,\ldots,\vec{u}_m}$ be the pre/postcondition parameters for all operators in $\opset$ \ST $\mat{C}[i][j] = \pclps{\vec{u}_i}{\vec{t}_j}$. 
Then:
\begin{itemize}
	\item if $\tperm$ is a producer symmetry that swaps $\vec{u}_i$ and $\vec{u}_j$, then $\cperm = \rowperm{i}{j}$,
	\item if $\tperm$ is a consumer symmetry that swaps $\vec{t}_i$ and $\vec{t}_j$, then $\cperm = \colperm{i}{j}$, and
	\item if $\tperm = \tperm_1 \permcomp \cdots \permcomp \tperm_k$ is a causal structure symmetry, then $\cperm = \cperm_1 \permcomp \cdots \permcomp \cperm_k$.
\end{itemize}
\end{defn}
%
A set of causal structure symmetries, that is, multi-row-column symmetries over $\mat{C}$, $\cgenerator$, can be broken by posting the lexicographic constraint $\MultiLex(\mat{C}, \cgenerator)$.
%
%%%%%%%%%%%%%%%%%%%%%%%%%%%%%%%%%%%%%%%%%%%%%%%%%%%%%%%%%%%%%%%%%%%%%%%%%%%%%%%%
\subsubsection{The \MRRC Encoding}
%%%%%%%%%%%%%%%%%%%%%%%%%%%%%%%%%%%%%%%%%%%%%%%%%%%%%%%%%%%%%%%%%%%%%%%%%%%%%%%%

\enlargethispage*{1\baselineskip}

The \MRRC encoding is an extension of \MRR that partially breaks causal structure symmetries with a set of hard \MAXSAT clauses that encode the \multilex constraint.
There are several steps to the construction of these clauses. 
First, the \POP's \PDG $\pdg_P$ is constructed, and $\generator$, the generating set of $\auts(\pdg_P)$, is computed. 
Then, $\generator$ is transformed into $\tgenerator$, a set of symmetries over $P$ \ST $\tperm \in \tgenerator$ \IFF $\perm \in \generator$ (Definition~\ref{def:aut-trans}). 
Finally, $\cgenerator$ is constructed, a set of multi-row-column symmetries over $\mat{C}$ \ST $\cperm \in \cgenerator$ \IFF $\tperm \in \tgenerator$ and $\tperm$ is a causal structure symmetry (Definitions~\ref{def:cs-symm} and~\ref{def:mrc-cs-symm}).
Formulae~\ref{eq:cs-aux-1}--\ref{eq:cs-lex-n} encode a set of hard \MAXSAT clauses propositionalise the \multilex constraint $\MultiLex(\mat{C}, \cgenerator)$:

\begin{align}
& \bigwedge_{\clapstack{ \cperm, \vec{\satprop}: \, \cperm \in \cgenerator, \\ \vec{\satprop} = \mat{C} \matfilter \tup{R_{\cperm}, C_{\cperm}}}} \quad \auxprop{\cperm}{1} \leftrightarrow (\vec{\satprop}[1] \leftrightarrow \cperm(\vec{\satprop}[1])). \label{eq:cs-aux-1} \\
	%
& \bigwedge_{\clapstack{ \cperm, \vec{\satprop}, i: \, \cperm \in \cgenerator, \\ \vec{\satprop} = \mat{C} \matfilter \tup{R_{\cperm}, C_{\cperm}}, \\ 2 \leq i < \card{\vec{\satprop}}}} \quad \auxprop{\cperm}{i} \leftrightarrow (\auxprop{\cperm}{i-1} \land \: (\vec{\satprop}[i] \leftrightarrow \cperm(\vec{\satprop}[i]))) . \label{eq:cs-aux-n} \\
	%
& \bigwedge_{\clapstack{ \cperm, \vec{\satprop}: \, \cperm \in \cgenerator, \\ \vec{\satprop} = \mat{C} \matfilter \tup{R_{\cperm}, C_{\cperm}}}} \quad \vec{\satprop}[1] \rightarrow \cperm(\vec{\satprop}[1]). \label{eq:cs-lex-1} \\
	%
& \bigwedge_{\clapstack{ \cperm, \vec{\satprop}, i: \, \cperm \in \cgenerator, \\ \vec{\satprop} = \mat{C} \matfilter \tup{R_{\cperm}, C_{\cperm}}, \\ 2 \leq i \leq \card{\vec{\satprop}}}} \quad \auxprop{\cperm}{i-1} \rightarrow \vec{\satprop}[i] \rightarrow \cperm(\vec{\satprop}[i]). \label{eq:cs-lex-n}
\end{align}
%
For each formula above, $\cperm \in \cgenerator$ is a multi-row-column symmetry over $\mat{C}$, and $\vec{\satprop} = \mat{C} \matfilter \tup{R_{\cperm}, C_{\cperm}}$ is the list of Boolean variables found on the left-hand side of the \multilex constraint as in Definition~\ref{def:multilex}.
Formulae~\ref{eq:cs-aux-1}--\ref{eq:cs-lex-n} encode a requirement that $\vec{\satprop} \lexleq \cperm(\vec{\satprop})$. 
The \textsc{AND-CSE} propositionalisation technique of \citet{Elgabou2015:Thesis} is used, which reduces the sizes of the formulae by replacing common subexpressions with auxiliary variables.

For each symmetry $\cperm$, the auxiliary variables $\auxprop{\cperm}{1},\ldots,\auxprop{\cperm}{n-1}$ are introduced, where $n = \card{\vec{\satprop}}$.
Formulae~\ref{eq:cs-aux-1} and~\ref{eq:cs-aux-n} state that auxiliary variable $\auxprop{\cperm}{i}$ is \true \IFF the first $i$ elements of $\vec{\satprop}$ and $\cperm(\vec{\satprop})$ are equal, that is, all \true or all \false.
Formulae~\ref{eq:cs-lex-1} and~\ref{eq:cs-lex-n} require that for all $\cperm$, if $i = 1$, or if all previous auxiliary propositions hold (i.e., all $\auxprop{\tperm}{j}$ such that $j < i$), then $\vec{\satprop}[i] \lexless \perm(\vec{\satprop}[i])$, that is, $\vec{\satprop}[i] \rightarrow \perm(\vec{\satprop}[i])$

The \MRRC encoding is an extension of \MRR that breaks causal structure symmetries with the above technique:
%
\begin{defn}\label{def:mrrc}
The \MRRC encoding transforms a \POP into a partially-weighted \MAXSAT instance through Formulae~\ref{eq:mrr-asymm}--\ref{eq:cs-lex-n} with the allowable orderings $\deoprec = \opset^2$.
\end{defn}
