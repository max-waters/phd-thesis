\section{Discussion}\label{sec:pop-maxsat-discussion}
%
The main contributions of this chapter were as follows:
%
\begin{itemize}
    \item This chapter introduced, and formally defined, the notions of \emph{reinstantiated reordering} and \emph{reinstantiated deordering}, transformations of a plan under which both ordering constraints and variable bindings can be changed.
    These ideas generalise the well-studied \emph{deordering} and \emph{reordering}, with the aim of finding optimised \POP{}s with fewer ordering constraints than would be possible had the bindings remained unchanged. 
    It was shown that finding an optimally reinstantiated \POP is \NP-hard and cannot be approximated within a constant factor.
    %
    \item A technique was presented for encoding the problem of finding a minimum reinstantiated de/reorder of a \POP into an instance of the partial weighted \MAXSAT problem. 
    This encoding is an extension and optimisation of \citet{Muise2016-PopMaxSAT}'s approach.
    %
    \item A number of symmetry breaking techniques were introduced to counter the exponential increase in search space that results from allowing variable bindings to change. 
    The notion of a \emph{causal structure symmetry} was introduced, a symmetry resulting from equivalent combinations of (potential) causal links between operator preconditions and postconditions. 
    To detect these symmetries, a \POP{}'s \emph{plan description graph} was defined, an undirected coloured graph that encodes relevant aspects of the \POP{}'s structure, with automorphisms that correspond to the \POP{}'s symmetries.
    %
    \item An empirical evaluation over all previous \IPC \STRIPS domains assessed the \MAXSAT-based reinstantiation approaches' ability to minimise a \POP{}'s ordering constraints, as compared to the polynomial, non-optimal but effective \EOG technique.
    Results show that in $55.6\%$ of cases, reinstantiation provides a $21.1\%$ \flex increase over \EOG.
    Interestingly, in nearly half of cases \EOG found a minimum reinstantiated reorder, indicating that the original choice of variables already allowed for optimal reordering.
    A comparison by planner showed that \dbfws benefits most from reinstantiation, with \MRRO providing a $31.6\%$ \flex increase, and that the optimised \madagascar plans are the most flexible, with an average \flex of $0.47$.
    While symmetry breaking does not result in a further increase in \flex, it does speed up search and allow the optimality of solutions to be exhaustively proved.
\end{itemize}
%
This chapter presented a practical technique for plan optimisation through the modification of both ordering and variable binding constraints.
Results show that in $55.6\%$ of cases, reinstantiation provides a \flex increase of $21.1\%$ over the \EOG baseline in $20.51\mins$, with results varying by domain.

The addition of symmetry breaking allows more plans to be optimally relaxed, but interestingly does not result in a further increase in \flex. 
This suggests that the encodings without symmetry breaking do find optimal solutions, but time out before this can be proved by the \MAXSAT solver.
Comparing the two symmetry breaking techniques shows that their success is dependent on domain, but it remains unclear what aspects of a domain allow a technique to succeed or not.
Determining this is complicated by the fact that solvers rely on variable ordering and branching heuristics that can clash with symmetry breaking constraints~\cite{Heller2008:ModelRestarts,Smith2005:SetsOfSbs}, and the choice of encoding of lex-leader constraints can significantly affect run time~\cite{Elgabou2015:Thesis}.

There are clear similarities between the causal structure symmetries introduced in this chapter and the commonly studied idea of \emph{strong stubborn sets} (Section~\ref{sec:order-symms}), a pruning technique in state-space planning that avoids exploring any two plans that are identical up to an interleaving of causally unrelated sequences of actions.
Both approaches examine possible causal connections between actions with the aim of reducing symmetry, and recognise that plans with equivalent causal structures produce redundancy in the search space.
However, beyond the obvious differences (stubborn sets are used by state-space planners to dynamically break symmetries, whereas causal structures are used to statically break symmetries when optimising \POP{}s), the two approaches also break different types of symmetry.
Stubborn sets are used to (broadly) prune classical plans with identical causal structures and actions, but symmetrical orderings, whereas the approach here is to prune \POP{}s with symmetrical causal structures, but (possibly) different actions (i.e., bindings) and orderings.

As reflected by the low coverage of reinstantiation-based encodings, the increase in \flex produced by allowing variable bindings to change comes with considerable computational cost.
Additionally, in a significant number of cases, the additional computation simply reveals that the original bindings were already optimal.

Whether \MRRO and \MRRC is preferable to less costly methods such as \EOG depends on the application. 
If offline preprocessing time is available and execution-time flexibility is paramount, or the application domain is one where reinstantiation can quickly and consistently improve plan flexibility (e.g., \ipcdomain{mystery}), then reinstantiation is clearly worthwhile.

This work generalises the \POP optimality definitions of \citet{Backstrom-CompAspects} and optimisation techniques of \citet{Muise2016-PopMaxSAT} to allow changes in variable bindings. 
As discussed in Section~\ref{sec:pop-opt-related-work}, other \POP optimisation techniques have been studied, for example those that de/reorder via block decomposition~\cite{SiddiquiHaslum2012-Block}, speed up search with \MILP models~\cite{Say2016:MathematicalPOP}, or optimise plan size or makespan~\cite{Muise2016-PopMaxSAT,Bercher2019:ElimActions,Bercher2020:POPvsPOCL,Say2016:MathematicalPOP}.
However, unlike the work presented here, these techniques do not allow variable bindings to change. 
Thus, further work could investigate whether generalising these techniques by allowing reinstantiation results in more flexible block decompositions or further reductions in plan size or makespan, and whether a \MILP encoding of \MRD, \MRR or their symmetry breaking extensions improves their execution time.
