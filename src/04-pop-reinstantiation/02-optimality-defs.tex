%%%%%%%%%%%%%%%%%%%%%%%%%%%%%%%%%%%%%%%%%%%%%%%%%%%%%%%%%%%%%%%%%%%%%%%%
\section{Optimality Criteria for Reinstantiated Partial-Order Plans}\label{sec:reins-pop-opt}
%%%%%%%%%%%%%%%%%%%%%%%%%%%%%%%%%%%%%%%%%%%%%%%%%%%%%%%%%%%%%%%%%%%%%%%%

The standard optimality criteria for \POP{}s\footnote{To more easily express changes in variable bindings, this chapter denotes a \POP as $P = \tup{\opset, \sub, \precrel}$, where $\opset$, $\sub$ and $\precrel$ denote the operators, bindings and orderings as in Definition~\ref{def:pop}.} (Definition~\ref{def:opt-de-reorder}) are based on a set-wise comparison of ordering constraints, and assume that the variable bindings in the \POP{}s remain static.
This section will present generalised versions of these optimality criteria which lift this restriction.
The aim of the new definitions remains the same, which is to compare the relative constrained-ness of two \POP{}s.
However, these extended definitions make use of additional information contained in the \POP, the variable bindings, and so are capable of comparing \POP{}s that would not be comparable under standard definitions.

First, the notions of deordering and reordering are extended to allow modifications to the \POP{}'s variable bindings.
A \emph{reinstantiated deordering} of a \POP removes ordering constraints and allows the variable bindings to change, while a \emph{reinstantiated reordering} allows arbitrary changes to both orderings and variable bindings.\footnote{There is no notion of ``deinstantiation'' as a \POP $P = \tup{\opset\sub, \prec}$ is always ground, i.e., bindings $\sub$ are ground and complete \WRT the operators \opset.}
In all cases, the resulting \POP must remain both ground (i.e., all variables are bound, as in Definition~\ref{def:pop}) and valid (i.e., all linearisations are executable, as in Definition~\ref{def:pop-validity}):\footnote{From Definition~\ref{def:pop}, these optimality criteria always compare the transitive closure of the \POP{}s' ordering constraints.}

\begin{defn}\label{def:reins-re-deorder} Let $P = \tup{\opset, \sub, \precrel}$ and $Q = \tup{\opset, \sub', \precrel'}$ be two \POP{}s. Then: 
\begin{itemize}
  \item $Q$ is a \defterm{reinstantiated reordering} of $P$ \IFF both $P$ and $Q$ are valid,
  \item $Q$ is a \defterm{reinstantiated deordering} of $P$ \IFF $Q$ is a reinstantiated reordering of $P$ and $\precrel' \subseteq \precrel$, and
  \item $Q$ is a \defterm{reinstantiated strict deordering} of $P$ \IFF $Q$ is a reinstantiated deordering of $P$ and $\precrel' \subset \precrel$. 
\end{itemize}
\end{defn}

For example, consider classical plans $P_1$ and $P_2$ and \POP $P_3$ in Figure~\ref{fig:rovers-plans}. 
As the three plans contain different actions (more specifically, the same operator types but different variable bindings), their flexibility cannot be compared under standard optimality definitions.
However, the generalised definitions above do allow a comparison: as all plans are valid and use the same operators, they are reinstantiated reorderings of each other, and $P_3$ is a reinstantiated (strict) deordering of both $P_1$ and $P_2$.

The notion of a least-constrained \POP can now be generalised to allow for changes in variable bindings.
A \emph{minimal reinstantiated deordering} of a \POP is a reinstantiated deordering that cannot be relaxed any further, that is, there is no modification to the \POP's variable bindings that will allow for the removal of any ordering constraints.
A \emph{minimum reinstantiated deordering} of a \POP has the fewest ordering constraints of all of its reinstantiated deorderings.
Out of all the ways that a \POP's ordering and binding constraints can be modified, a \emph{minimum reinstantiated reordering} contains the fewest ordering constraints:

\begin{defn}\label{def:opt-reins-re-deorder} Let $P = \tup{\opset, \sub, \precrel}$ and $Q = \tup{\opset, \sub', \precrel'}$ be two \POP{}s. Then: 
\begin{itemize}
  \item $Q$ is a \defterm{minimal reinstantiated deordering} of $P$ \IFF it is a reinstantiated deordering of $P$ and there is no \POP $R$ such that $R$ is a reinstantiated strict deordering of $Q$,
  \item $Q$ is a \defterm{minimum reinstantiated deordering} of $P$ \IFF it is a reinstantiated deordering of $P$ and there is no \POP $R = \tup{\opset, \sub'', \precrel''}$ such that $R$ is a reinstantiated deordering of $P$ and $\card{\precrel''} < \card{\precrel'}$, and 
  \item $Q$ is a \defterm{minimum reinstantiated reordering} of $P$ \IFF it is a reinstantiated reordering of $P$ and there is no \POP $R = \tup{\opset, \sub'', \precrel''}$ such that $R$ is a reinstantiated reordering of $P$ and $\card{\precrel''} < \card{\precrel'}$.
\end{itemize}
\end{defn}

For example, plan $P_3$ has two ordering constraints, and as there is no reinstantiation of $P_1$ which allows for fewer than this, $P_3$ is a minimum reinstantiated reordering of $P_1$.


%%%%%%%%%%%%%%%%%%%%%%%%%%%%%%%%%%%%%%%%%%%%%%%%%%%%%%%%%%%%%%%%%%%%%%%%
\subsection{Computing Optimal Reinstantiations}
%%%%%%%%%%%%%%%%%%%%%%%%%%%%%%%%%%%%%%%%%%%%%%%%%%%%%%%%%%%%%%%%%%%%%%%%

Given that the size of a \POP's ordering relation can be measured directly, and that reinstantiated de/reordering is a generalisation of standard de/reordering, it is no surprise that deciding whether a \POP has a reinstantiated de/reordering with fewer than $k$ ordering constraints is \NP-complete:\footnote{Proofs for all theorems in this chapter are in Appendix~\ref{apx:pop-maxsat}.}

\begin{restatable}{theorem}{mmrd}\label{thrm:mrd-mrr-npc}
\csprob{Minimum Reinstantiated Reordering}
Given a \POP $P = \tup{\opset, \sub, \precrel}$ and an integer $k > 0$, determining whether there exists a \POP $Q = \tup{\opset, \sub', \precrel'}$ \ST $Q$ is a reinstantiated reordering (or deordering) of $P$ and $\card{\precrel'} < k$ is \NP-complete.
\end{restatable}

Additionally, the optimisation problems of finding a minimum reinstantiated de/reordering are \NP-hard and cannot be approximated within a constant factor:

\begin{restatable}{theorem}{mmrdopt}\label{thrm:mrd-mrr-opt}
\csprob{Approximate Minimum Reinstantiated Reordering}
The problem of finding a minimum reinstantiated deordering (or reordering) of a \POP is not in \APX unless $\NP \subseteq \DTIME(n^{\text{poly log n}}$).
\end{restatable}
%
\paragraph{Optimality Under \POCL-validity} The results and definitions above use the standard definition of \POP validity, that requires that all of a \POP{}'s linearisations be executable (Definitions~\ref{def:pop-lin} and~\ref{def:pop-validity}).
Under the additional requirement that the input and optimised \POP{}s be \POCL-valid, that is, that each operator precondition be supported by a postcondition of some previous operator (Definition~\ref{def:pocl-valid}), minimum de/reordering remain \NP-complete (a trivial corollary of Theorem $4.8$ in~\cite{Backstrom-CompAspects}). 
However, due to the stronger requirements of \POCL-validity there exist minimum \POCL-valid de/reorderings that can be further optimised while remaining valid under Definition~\ref{def:pop-validity}~\cite{Kambhampati1996:ModalTruth}.
These complexity and completeness results can be trivially extended to minimum reinstantiated \POCL-valid de/reordering.