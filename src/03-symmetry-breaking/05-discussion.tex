\section{Discussion}\label{sec:symm-breaking-discussion}
%
The main contributions of this chapter were as follows:
%
\begin{itemize}
    \item This chapter formally defined a hierarchy of generalised symmetries in graph and matrix models.
    While standard approaches to symmetry breaking in matrix models break \emph{row symmetries} and \emph{column symmetries}, in which rows and columns are individually interchangeable, this chapter identified \emph{multi-row symmetries} and \emph{multi-column symmetries}, in which combinations of rows and columns are interchangeable but individual rows and columns are not, and \emph{multi-row-column symmetries}, in which combinations of rows and columns are interchangeable.
    \item Similarly generalised symmetries in graph models, termed \emph{multi-vertex symmetries}, were defined and shown to be special cases of multi-row-column symmetries.
    \item A series of examples show that these symmetries can occur in many plan optimisation problems and \CSP benchmark domains~\cite{csplib}.
    \item As these symmetries cannot be broken with standard techniques such as \dblex~\cite{Flener2002:DoubleLex} and \snakelex~\cite{Grayland2009:SnakeLex}, a novel symmetry breaking constraint was introduced, \multilex, that generalises \dblex and can break generalised symmetries in matrix and graph models, including directed graphs with self-loops.
    \item An empirical evaluation was performed over a series of unconstrained matrix, simple graph and \DAG models of different sizes and with different symmetries. 
    Results show that \myi, despite being weaker than row and column symmetries, multi-row-column symmetries still result in a factorial size increase in the search space, and \myii, \multilex is a compact and efficient symmetry breaking constraint that removes all but a constant factor of non-canonical solutions.
\end{itemize}
%
Sections~\ref{sec:multi-lex}~and~\ref{sec:multi-lex-evaluation} establish that \multilex is capable of efficiently breaking a range of generalised symmetries in matrix and graph models of all kinds.
While this allows for an exploitation of symmetries not possible by previous means, and with no additional constraint size cost, a drawback is that the detection of multi-row-column symmetries can be a complex task.

While row and column symmetries are often obvious from the domain description, or can be found by a simple analysis of a problem instance, detecting multi-row-column symmetries can require specialised graph automorphism software such as \NAUTY~\cite{McKay2104:PracGraphIso}.
For example, in the \cspdomain{Social Golfers} problem (\csplib problem~$10$), a weekly golf schedule must be generated where players play together no more than once. 
It is clear from the domain definition that the times, groups and golfers are all interchangeable, and so any problem instance will have total row and column symmetry.
Symmetries in the \cspdomain{Progressive Party} problem (\csplib problem~$13$), in which yacht parties must be scheduled such that crews meet no more than once and boat capacities are respected, are not immediate from the domain description. 
However, detecting them in a problem instance is a simple matter of finding all boats with identical capacities and crew sizes.

In contrast, the examples in Section~\ref{sec:multi-row-col-symm} show that finding multi-row-column symmetries requires finding equivalent \emph{combinations} of objects in the problem instance.
The standard approach~\cite{Crawford1996:SymmetryPredicates} of reducing this complex task to the graph automorphism problem (Sections~\ref{sec:automorphisms}~and~\ref{sec:symm-det-break}) requires (possibly domain-specific) techniques for generating coloured graph representations of problem instances.
This technique, while well established~\cite{Puget2005:AutoDetSymm,Mears2009:ImplSymmDetection,JoslinRoy97:ExploitingSymm,Pochter2011:SymmStatePlanners}, is not trivial.

A notable exception to this is the \cspdomain{Vessel Loading} problem (Figure~\ref{fig:vessel-example}), in which vertical and horizontal reflection symmetries are immediate from the domain description, and, interestingly, can be expressed as multi-row and multi-column symmetries, respectively.
An obvious further consideration, as to whether the rotational symmetries exhibited by domains such as \cspdomain{N-Queens} (\csplib problem~$54$) can also be handled by \multilex, reveals the limitations of the generalised symmetry definitions in Section~\ref{sec:multi-row-col-symm}.
Multi-row-column symmetries swap rows with rows and columns with columns, however, a rotational symmetry swaps row with columns.
Thus, further work could explore \multilex as a novel approach to statically breaking reflection symmetries, or extend it to break rotational symmetries.

The results in Section~\ref{sec:multi-lex-evaluation}, in which the exponential expansion of the search space created by multi-row-column symmetries was nearly entirely removed by \multilex, suggest that the additional work required to detect generalised symmetries is worthwhile.
However, further work is required to determine whether these results can be replicated under the more complex constraint structures of standard benchmark problems.

Section~\ref{sec:symm-results-ml-eff} proposed an asymptotic limit on \multilex's efficiency. 
While this was justified empirically, an analytical justification would shed light on \multilex's, and therefore \dblex's, behaviour, possibly leading to more efficient symmetry breaking techniques.