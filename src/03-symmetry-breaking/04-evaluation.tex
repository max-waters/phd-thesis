\pagebreak
\section{Experimental Evaluation}\label{sec:multi-lex-evaluation}

This evaluation addresses two questions.
As discussed in Section~\ref{sec:gen-symm-hierarchy}, more general symmetry groups can express ``weaker'' symmetries.
The first question is thus whether multi-row-column symmetries introduce a significant number of non-canonical solutions into the search space, as compared with stronger symmetries that swap the same rows and columns individually.
Secondly, \multilex is a partial symmetry breaking constraint, raising the question of its effectiveness in practice. 

Symmetry breaking techniques are typically assessed by measuring the degree to which they improve search times, however, this approach is not ideal.
Constraint solvers rely on variable ordering and branching heuristics that can clash with symmetry breaking constraints~\cite{Heller2008:ModelRestarts,Smith2005:SetsOfSbs}, meaning that any difference in search time cannot be attributed solely to symmetry reduction.
Additionally, isolating the degree of symmetry introduced by multi-row-column symmetries is complicated by the fact that benchmark problems may contain a combination of generalised and simple symmetries, or no generalised symmetries at all.
Thus, this section takes the same approach as~\citet{Shlyakhter2007:SymmBreaking}, and answers the above questions by counting the number of non-canonical solutions that are introduced by generalised symmetries, and removed by \multilex, over a range of synthetic test cases.
%, that is, unconstrained matrix, simple graph and \DAG models of different sizes and with different symmetries.
%
%%%%%%%%%%%%%%%%%%%%%%%%%%%%%%%%%%%%%%%%%%%%%%%%%%%%%%%%%%%%%%%%%%%%%%%%%%%%%%%%
\subsection{Experimental Setup}
%%%%%%%%%%%%%%%%%%%%%%%%%%%%%%%%%%%%%%%%%%%%%%%%%%%%%%%%%%%%%%%%%%%%%%%%%%%%%%%%
%
Tests were performed over unconstrained matrix models, simple graph models and \DAG models with a range of sizes and symmetries.
Each test case comprises a variable matrix $\mat{M}$, a set of constraints $\mat{C}$, and a symmetry group $\group$ with generator $\generator$. 
For all matrix models $\mat{C} = \emptyset$, and for graph models $\mat{C}$ contains only those constraints required to ensure that a solution represents a simple graph or \DAG, as appropriate. 
For each model, a range of different symmetry groups were created by dividing the rows and columns into varying numbers of equal-sized divisions as in Definition~\ref{def:multi-row-col-symm-grp}.
Four sets of tests were performed, as described below:
%
%%%%%%%%%%%%%%%%%%%%%%%%%%%%%%%%%%%%%%%%%%%%%%%%%%%%%%%%%%%%%%%%%%%%%%%%%%%%%%%%
\paragraph{A: Matrices with multi-row-column symmetry} Test cases comprise an unconstrained $r \times c$ matrix model $\mat{M}$ and a multi-row-column symmetry group $\group$ with generator $\generator$ defined by partitioning the rows and columns into $n$ interchangeable divisions.
%For example, a $4 \times 2$ matrix with $n = 2$ indicates that $\group$ and $\generator$ are defined from the row-column divisions $\tup{\tup{\mat{M}[1], \mat{M}[2]}, \tup{\mat{M}[][1]}}$ and $\tup{\tup{\mat{M}[3], \mat{M}[4]}, \tup{\mat{M}[][2]}}$.
Tests cover all $2 \leq c \leq r \leq 6$ and all $n$ \ST $n | r$ and $n | c$.\footnote{$x | y$ indicates that $x$ and $y$ are positive integers and $y$ is a multiple of $x$.}
%The values of $\cancount$ and $\mlcount$ were computed by counting the satisfying assignments to BLAH and BLAH, respectively.
%
%%%%%%%%%%%%%%%%%%%%%%%%%%%%%%%%%%%%%%%%%%%%%%%%%%%%%%%%%%%%%%%%%%%%%%%%%%%%%%%%
\paragraph{B: Matrices with multi-row and multi-column symmetry} Test cases comprise an unconstrained $r \times c$ matrix model $\mat{M}$ and a multi-row and multi-column symmetry group $\group = \group_r \times \group_c$ with generator $\generator = \generator_r \cup \generator_c$.
Multi-row symmetry group $\group_r$ and multi-column symmetry group $\group_c$ were produced by partitioning the rows into $n_r$ interchangeable lists, the columns into $n_c$ interchangeable lists, respectively.
%For example, a $4 \times 2$ matrix $\mat{M}$ with $n_r = n_c = 3$, indicates that $\group_R$ and $\generator_R$ are defined from the row-column divisions $\tup{\tup{\mat{M}[1], \mat{M}[2]}, \tup{}}$ and $\tup{\tup{\mat{M}[3], \mat{M}[4]}, \tup{}}$, and $\group_C$ and $\generator_C$ are defined from $\tup{\tup{}, \tup{\mat{M}[][1]}}$, $\tup{\tup{}, \tup{\mat{M}[][2]}}$.
Tests cover all $2 \leq c \leq r \leq 6$ and all $n_r, n_c$ \ST $n_r | r$ and $n_c | c$.
%When $n_r = r$ and $n_c = c$ (i.e., all rows and columns are interchangeable), $\cancount$ was taken from \citet{Harisson1973:BinaryMatrices}, otherwise it was computed as described above.
%
%%%%%%%%%%%%%%%%%%%%%%%%%%%%%%%%%%%%%%%%%%%%%%%%%%%%%%%%%%%%%%%%%%%%%%%%%%%%%%%%

\paragraph{C: Simple graph models} Test cases comprise a $v$-vertex simple graph model $\mat{M}$ and a multi-vertex symmetry group produced by dividing the vertices into $n$ interchangeable lists.
%For example, if $v = 4$ and $n = 2$, then $\group$ and $\generator$ are defined from the row-column divisions $\tup{\tup{\mat{M}[1], \mat{M}[2]}, \tup{\mat{M}[1], \mat{M}[2]}}$ and $\tup{\tup{\mat{M}[3], \mat{M}[4]}, \tup{\mat{M}[3], \mat{M}[4]}}$.
Constraint set $\mat{C}$ ensures that solutions represent simple graphs:
%
\begin{align}
    \bigwedge_{1 \leq i \leq v} \neg & \mat{M}[i][i]. \label{eq:simple-1} \\
    \bigwedge_{1 \leq i < j \leq v} & \mat{M}[i][j] \leftrightarrow \mat{M}[j][i]. \label{eq:simple-2}
\end{align}
%
Tests cover all graph models with all $2 \leq v \leq 12$, and all $n$ \ST $n | v$.
%When $v = n$ (i.e., all vertices are interchangeable), $\cancount$ was taken from \citet{Sloane1973:Handbook}, otherwise it was computed as described above.
%
%%%%%%%%%%%%%%%%%%%%%%%%%%%%%%%%%%%%%%%%%%%%%%%%%%%%%%%%%%%%%%%%%%%%%%%%%%%%%%%%
\paragraph{D: \DAG models} Test cases comprise a $v$-vertex \DAG model and a multi-vertex symmetry group defined by dividing the vertices into $n$ interchangeable lists.
The \DAG model contains a $v^2$ adjacency matrix $\mat{M}$, and additional variables and constraints to enforce acyclicity.
The $v^2$ matrix $\mat{T}$ contains the transitive closure of $\mat{M}$, with $\mat{T}[i][j] = 1$ indicating that $\gvert_j$ is reachable from $\gvert_i$, and $\mat{A}$ is a $v^3$ matrix with $\mat{A}[i][j][k]$ indicating that $\gvert_k$ is reachable from $\gvert_i$ via $\gvert_j$.
Formulae~\ref{eq:dag-1} and~\ref{eq:dag-2} define the values of $\mat{T}$ and $\mat{A}$, and Formula~\ref{eq:dag-3} enforces acyclicity:
%
\begin{align}
    & \bigwedge_{1 \leq i,j,k \leq v} \mat{A}[i][j][k] \leftrightarrow \mat{M}[i][j] \land \mat{T}[j][k]. \label{eq:dag-1} \\
    & \bigwedge_{1 \leq i,j \leq v} \mat{T}[i][j] \leftrightarrow ( \mat{M}[i][j] \lor \bigvee_{1 \leq k \leq n} \mat{A}[i][k][j] ). \label{eq:dag-2} \\
    & \bigwedge_{1 \leq i \leq v} \neg \mat{T}[i][i]. \label{eq:dag-3}
\end{align}
%
Tests cover all \DAG models with $2 \leq v \leq 8$, and all $n$ \ST $n | v$.
%The number of solutions $\solcount$ was taken from \citet{Harary1973:GraphEnum}, as was $\cancount$ when $n = v$ (i.e., all vertices are interchangeable).
%Otherwise, $\cancount$ was computed by counting the solutions as described above.
%

\bigskip

For each test case, the total number of solutions ($\solcount$, i.e., the total number of $r \times c$ matrices, or simple or directed acyclic graphs with $v$ vertices), the number of canonical solutions ($\cancount$) and the number of solutions allowed by \multilex ($\mlcount$) are reported.
All values of $\solcount$ are well-known~\cite{Harary1973:GraphEnum}, and some values of $\cancount$~\cite{Harisson1973:BinaryMatrices,Sloane1973:Handbook,Harary1973:GraphEnum}.
Unknown values of $\cancount$ were computed by counting the solutions that satisfy $\mat{C} \land \bigwedge_{\perm \in \group} \mat{M}_R \lexleq \perm(\mat{M}_R)$, and the value of $\mlcount$ was computed by counting the solutions that satisfy $\mat{C} \land \MultiLex(\mat{M}, \generator)$.
For this, the exact model counters \relsat~\cite{Bayardo200:CountingModels} and \ganak~\cite{Sharma2019:Ganak} were used over appropriately constructed \SAT instances.

The degree of symmetry created by a symmetry group is measured by the proportion of non-canonical solutions, that is, $\symmrate = (\solcount - \cancount) / \solcount$, and the success of \multilex is measured by its efficiency in ruling out non-canonical solutions, that is $\eff = (\solcount - \mlcount) / (\solcount - \cancount)$.

%%%%%%%%%%%%%%%%%%%%%%%%%%%%%%%%%%%%%%%%%%%%%%%%%%%%%%%%%%%%%%%%%%%%%%%%%%%%%%%%
\subsection{Results}
%%%%%%%%%%%%%%%%%%%%%%%%%%%%%%%%%%%%%%%%%%%%%%%%%%%%%%%%%%%%%%%%%%%%%%%%%%%%%%%%
%
\subimport*{./tables/}{results.tex}
%
Results are displayed in Tables~\ref{tab:mat-mrc-results}--\ref{tab:graph-dag-results}.
Canonical solution counts indicate that while generalised symmetries result in fewer redundant solutions than row, column or vertex symmetries, they still result in a factorial size increase in the search space.
Efficiency results show that \multilex removes nearly all non-canonical solutions, and that as the model size increases, the number of solutions that remain in each symmetry class tends towards a constant factor of the input.
%
%%%%%%%%%%%%%%%%%%%%%%%%%%%%%%%%%%%%%%%%%%%%%%%%%%%%%%%%%%%%%%%%%%%%%%%%%%%%%%%%
\subsubsection{Symmetry Rates for Generalised Symmetries}
%
As expected, generalised symmetries introduce fewer symmetrical solutions than their simple counterparts.
For example, Table~\ref{tab:mat-mrmc-results} shows that while a $4 \times 4$ matrix with total row and column symmetry ($n_r = n_c = 4$) has $317$ canonical solutions ($99.51\%$ symmetry), splitting the row and columns into two interchangeable groups of two ($n_r = n_c = 2$) increases this to $16,576$ ($74.7\%$ symmetry).
More extreme differences occur in larger models.
For example, Table~\ref{tab:graph-simple-results} shows that a $12$-vertex simple graph with total vertex symmetry ($n=12$) has $1.6 \times 10^{11}$ canonical solutions ($99.99\%$ symmetry), but splitting the vertices into two interchangeable groups of six ($n=2$) increases this to $3.6 \times 10^{19}$ ($49.99\%$ symmetry).
Nevertheless, generalised symmetries render a significant proportion of the search space redundant, with $\symmrate \geq 95\%$ for most non-trivial matrices. 

\subimport*{./tables/}{canon-rate-plot.tex}
%
Closer examination reveals that the proportion of non-canonical solutions increases factorially with the number of interchangeable divisions of rows and/or columns, or vertices.
This should come as no surprise.
\citet{Polya1937:Enumeration} observed that the proportion of $v$-vertex graphs that admit only trivial automorphisms (i.e., $\perm(G) = G$ \IFF $\perm = \idperm$) approaches $1$ as $v$ increases. 
Thus, for any given $\group$, the number of symmetric alternatives for a graph will tend towards $\card{\group}-1$, and the proportion of graphs that are canonical \WRT $\group$ tends towards $1/\card{\group}$.

To determine how quickly this theoretical limit is reached in the context of simple graphs, \DAG{}s and matrices (i.e., undirected bipartite graphs) with generalised symmetries, the value $\Delta_c = (\cancount/\solcount) - (1/\card{\group})$ was computed for each test case, that is, the difference between the observed proportion of canonical solutions and the predicted asymptotic limit.
As multi-vertex and multi-row-column symmetry groups are isomorphic to $S_n$ and multi-row and multi-column symmetry groups are isomorphic to $S_{n_r} \times S_{n_c}$ (Section~\ref{sec:gen-symm-grps}), this limit is $1/n!$ and $1/(n_r!n_c!)$, respectively.

Results show that $\Delta_c$ approaches $0$ as either the model size $\card{\mat{M}}$ increases or $n$ decreases.
Figure~\ref{fig:can-rate-plot} depicts this trend by plotting $\card{\mat{M}}$ against $\Delta_c$ for all test cases with values of $n, n_r, n_c \leq 4$ for which multiple data points are available. 
(For all test cases not included in the graph, $\Delta_c \leq 0.016$.)
The graph indicates the proportion of canonical solutions quickly approaches $1/\card{\group}$.
Indeed, for all matrix models with $r \times c \geq 15$, and graph models with $v \geq 6$, $\Delta_c \leq 0.01$.

%%%%%%%%%%%%%%%%%%%%%%%%%%%%%%%%%%%%%%%%%%%%%%%%%%%%%%%%%%%%%%%%%%%%%%%%%%%%%%%%
\subsubsection{Effectiveness of \multilex}\label{sec:symm-results-ml-eff}
%
The results in Tables~\ref{tab:mat-mrc-results}--\ref{tab:graph-dag-results} show that despite being a partial symmetry breaking constraint, \multilex consistently removes nearly all non-canonical solutions. 
Its efficiency drops below $95\%$ in just nine of $94$ test cases, and all symmetries are broken in $19$.

Closer examination reveals that for multi-vertex and multi-row-column symmetry groups, the proportion of solutions allowed by \multilex, $\mlcount/\solcount$, approaches $1/2(n-1)!$ as the model size increases.
(Any pattern in matrix models with multi-row and multi-column symmetry was not immediately clear).
Unlike the trend shown in Figure~\ref{fig:can-rate-plot}, this value is not justified theoretically, but rather by an empirical analysis that shows it to be an accurate predictor of \multilex's efficiency.

\pagebreak

For all test cases over multi-vertex and multi-row-column symmetries, Figure~\ref{fig:ml-all-plot} plots $\mlcount/\solcount$, the proportion of solutions left by \multilex, against $1/2(n-1)!$, the predicted asymptotic limit. 
A clear correlation is shown, with $\mlcount/\solcount \leq 1/2(n-1)!$ in all cases.
Overall, predicting $\mlcount/\solcount$ with $1/2(n-1)!$ gives a root mean squared error of $0.09$, and when restricted to matrices with multi-row-column symmetries, simple graphs or \DAG{}s it is $0.04$, $0.12$ and $0.06$, respectively.

Prediction errors decrease as the model size $\card{\mat{M}}$ increases and $n$ decreases.
Figure~\ref{fig:ml-rate-plot} shows this trend by plotting $\card{\mat{M}}$ against $\Delta_m = (\mlcount/\solcount) - (1/2(n-1)!)$ over values of $n \leq 4$ for which multiple data points are available. 
(For all test cases not included in the graph, $\Delta_m \leq 0.018$.)
All matrix models with $r \times c \geq 12$ are within $0.02$ of the asymptotic limit, all simple graphs and \DAG{}s with $v \geq 6$ are within $0.01$ and $0.04$, respectively.

This observed trend indicates that \multilex performs better than could be expected.
\citet{Katsirelos:ComplComplSymmBr} demonstrate that in the worst case, \dblex, and therefore \multilex, leaves an exponential number of non-canonical solutions in each symmetry class.
However, the more general asymptotic limits described above show that as the matrix size increases, this number tends towards a constant factor of $n$.
Since the number of symmetry groups (i.e., canonical solutions) tends towards $\solcount/n!$, and the number of solutions allowed by \multilex tends towards $\solcount/2(n-1)!$, it follows that the number of non-canonical solutions per symmetry class that are not removed by \multilex tends towards $n!/2(n-1)! = n/2$.
