%%%%%%%%%%%%%%%%%%%%%%%%%%%%%%%%%%%%%%%%%%%%%%%%%%%%%%%%%%%%%%%%%%%%%%%%%%%%%%%%
\section{Generalised Symmetries in Matrix and Graph Models}\label{sec:multi-row-col-symm}
%%%%%%%%%%%%%%%%%%%%%%%%%%%%%%%%%%%%%%%%%%%%%%%%%%%%%%%%%%%%%%%%%%%%%%%%%%%%%%%%
%
%Broadly speaking, a symmetry is a solution-preserving automorphism of a problem description (Section~\ref{sec:symm-det-break}).
In the context of \emph{matrix models} (i.e., \CSP{}s containing one or more matrices of variables), a \emph{variable symmetry} is a permutation of variables that leaves the model's constraints unchanged, and preserves the validity and optimality of its solutions (Section~\ref{sec:symm-det-break}).
Arguably the most common forms of symmetry in matrix models are \emph{row symmetries} and \emph{column symmetries}, special cases of variable symmetry that result from individually interchangeable domain objects, and swap a pair of rows or columns of variables, respectively.
%Standard approaches to symmetry breaking target individually interchangeable rows or columns, and cannot be applied to more general symmetries. 
These can be generalised into \emph{multi-row symmetries} and \emph{multi-column symmetries}, in which multiple rows or columns are simultaneously swapped, and \emph{multi-row-column symmetries}, which, as shown in Section~\ref{sec:symm-intro-ex} above, swap multiple pairs of rows and columns.

As \emph{graph models} (i.e., \CSP{}s that model the adjacency matrix of a graph) associate vertices with both a row and column of an adjacency matrix, vertices $\gvert_i$ and $\gvert_j$ being interchangeable results in a special case of multi-row-column symmetry that swaps rows $i$ and $j$, and columns $i$ and $j$.
These symmetries, termed here \emph{vertex symmetries}, can be naturally generalised to \emph{multi-vertex symmetries}.
This section will formally define this symmetry hierarchy, and for each type of symmetry, a corresponding category of symmetry group (Section~\ref{sec:permutations}).

As these symmetries are not confined to plan optimisation problems--a brief survey of \csplib reveals that they can occur in many matrix and graph-based \CSP domains--illustrative examples will be taken from combinatorial and graph theoretic problems.
In all examples, a matrix of variables exhibits symmetries in which combinations of rows and/or columns are interchangeable, but individual rows and columns are not.
%
%%%%%%%%%%%%%%%%%%%%%%%%%%%%%%%%%%%%%%%%%%%%%%%%%%%%%%%%%%%%%%%%%%%%%%%%%%%%%%%%
\subsection{Multi-Row Symmetries and Multi-Column Symmetries}
%%%%%%%%%%%%%%%%%%%%%%%%%%%%%%%%%%%%%%%%%%%%%%%%%%%%%%%%%%%%%%%%%%%%%%%%%%%%%%%%
%
All of the symmetry types outlined above swap entire rows and/or columns of variables.
To make their formal definitions more concise, the notation $\rowperm{i}{j}$ and $\colperm{i}{j}$ is used to denote a permutation that exchanges the variables in a pair of rows or columns, respectively, and fixes all others: 
%
\begin{defn}\label{def:row-col-perm} Let $\mat{M}$ be an $r \times c$ matrix. Then:
\begin{itemize}
    \item for $1 \leq i, j \leq r$, $\rowperm{\mat{M},i}{j} = \cycle{\mat{M}[i][1], \; \mat{M}[j][1]}$ $\cdots$ $\cycle{\mat{M}[i][c], \; \mat{M}[j][c]}$, and
    \item for $1 \leq i, j \leq c$, $\colperm{\mat{M},i}{j} = \cycle{\mat{M}[1][i], \; \mat{M}[1][j]}$ $\cdots$ $\cycle{\mat{M}[r][i], \; \mat{M}[r][j]}$.
\end{itemize}
\end{defn}
%
When clear from context, $\mat{M}$ is excluded from the notation.
For example, if $\mat{M} = \tup{\tup{x_1,x_2,x_3},\tup{x_4,x_5,x_6},\tup{x_7,x_8,x_9}}$ is a $3^2$ matrix, then $\rowperm{1}{2} = \cycle{x_1,x_4}\cycle{x_2,x_5}\cycle{x_3,x_6}$ is a permutation that swaps each variable in the first row with the corresponding variable in the second.
Note that in all cases, $\rowperm{i}{j} = \rowperm{j}{i}$ and $\colperm{i}{j} = \colperm{j}{i}$, and when $i = j$, $\rowperm{i}{j} = \colperm{i}{j} = \idperm$.

\emph{Row symmetries} and \emph{column symmetries} are special cases of variable symmetry that swap pairs of rows or columns, respectively:
%
\begin{defn}\label{def:row-col-symm} Let $\mat{M}$ be a matrix with symmetry $\perm$. Then: 
\begin{itemize}
    \item $\perm$ is a \defterm{row symmetry} \IFF there exists an $i, j$ \ST $\perm = \rowperm{i}{j}$, and
    \item $\perm$ is a \defterm{column symmetry} \IFF there exists an $i, j$ \ST $\perm = \colperm{i}{j}$.
\end{itemize}
\end{defn}
%
A \emph{multi-row symmetry} is a generalised row symmetry that swaps multiple pairs of rows simultaneously. 
It is expressible as the composition of multiple row symmetries, with the condition that any given row is swapped at most once.
%For example, if $\perm_1$ swaps rows $1$ and $2$, and $\perm_2$ swaps rows $3$ and $4$, then $\perm = \perm_1 \permcomp \perm_2$ is a row symmetry; however, if $\perm_2$ swaps rows $2$ and $3$, then it is not.
Column symmetries can be similarly generalised into \emph{multi-column symmetries}:
%
\begin{defn}\label{def:multi-row-multi-col-symm} Let $\mat{M}$ be a matrix with symmetry $\perm$. Then:
\begin{itemize}
    \item $\perm$ is a \defterm{multi-row symmetry} \IFF there exist two disjoint sets of row indices $i_1,\ldots,i_n$ and $j_1,\ldots,j_n$ \ST $\perm = \rowperm{i_1}{j_1} \permcomp \cdots \permcomp \rowperm{i_n}{j_n}$, and
    %
    \item $\perm$ is a \defterm{multi-column symmetry} \IFF there exist two disjoint sets of column indices $i_1,\ldots,i_n$ and $j_1,\ldots,j_n$ \ST $\perm = \colperm{i_1}{j_1} \permcomp \cdots \permcomp \colperm{i_n}{j_n}$.
\end{itemize}
\end{defn}
%
\subimport*{graphics/}{vessel-example.tex}

For example, the goal of the \cspdomain{Vessel Loading} problem (\csplib problem~$8$), is to find an arrangement of crates on a rectangular deck that satisfies distance constraints. 
An instance can be modelled as a matrix $\mat{D}$ containing the crate type at each location. 
Figure~\ref{fig:vessel-example} shows three symmetrical solutions to a simple example in which two crates of the same type must be placed in non-adjacent locations on a $2 \times 4$ deck.  
As reflecting the deck about its horizontal or vertical axis will produce an equally valid solution, this instance has a row symmetry $\rowperm{1}{2}$ that swaps the rows, and a multi-column symmetry $\colperm{1}{4} \permcomp \colperm{2}{3}$ that simultaneously swaps columns one and two with columns four and three, respectively.
%From Definition~\ref{def:multi-row-multi-col-symm}, $\perm_2$ is expressible as $\perm_{c_1} \permcomp \perm_{c_2}$, where $\perm_{c_1}$ swaps columns one and four, and $\perm_{c_2}$ swaps columns two and three.
Solutions $\mat{D}_2$ and $\mat{D}_3$ are the result of applying these two symmetries, respectively, to $\mat{D}_1$.

Other \CSP domains with multi-row and/or multi-column symmetry include \cspdomain{Balanced Academic Curriculum}, \cspdomain{Wagner-Whitin Distribution}, \cspdomain{N-Queens} (and its variations), \cspdomain{Travelling Tournament} and \cspdomain{Layout} (\csplib problems $30$, $40$, $54$, $68$ and $132$).
%
%%%%%%%%%%%%%%%%%%%%%%%%%%%%%%%%%%%%%%%%%%%%%%%%%%%%%%%%%%%%%%%%%%%%%%%%%%%%%%%%
\subsection{Muli-Row-Column Symmetries}
%%%%%%%%%%%%%%%%%%%%%%%%%%%%%%%%%%%%%%%%%%%%%%%%%%%%%%%%%%%%%%%%%%%%%%%%%%%%%%%%
%
A \emph{multi-row-column symmetry} is a further generalisation that simultaneously swaps multiple rows and columns.
These symmetries can be expressed as the composition of a multi-row symmetry and a multi-column symmetry:

\begin{defn}\label{def:multi-row-col-symm} If $\mat{M}$ is a matrix with symmetry $\perm$, then $\perm$ is a \defterm{multi-row-column symmetry} \IFF it is expressible as the composition of a multi-row symmetry and a multi-column symmetry.
\end{defn}
%
\subimport*{graphics/}{winner-example.tex}
%
For example, the \cspdomain{Winner Determination} problem (\csplib problem~$53$) simulates an auction in which numeric bids are placed for (possibly overlapping) sets of items, and the aim is to find a set of disjoint offers that maximises the sale price.
An instance with $r$ bids over $c$ items can be modelled as an $r \times c$ matrix where $\mat{W}[i][j] = 1$ \IFF bid $b_i$ has been accepted for item $t_j$.
Figure~\ref{fig:winner-example} depicts a small example comprising two bids over four items, and two optimal solutions.
As swapping symbols $b_1$, $t_1$ and $t_2$ with $b_2$, $t_3$ and $t_4$, respectively, in Figure~\ref{fig:winner-example-bids} leaves the instance unchanged, this instance has a multi-row-column symmetry $\perm = \rowperm{1}{2} \permcomp \colperm{1}{3} \permcomp \colperm{2}{4}$.
Applying $\perm$ to $\mat{W}_1$ produces the equally valid and optimal $\mat{W}_2$.

Other matrix models with multi-row-column symmetry include \cspdomain{Warehouse Allocation}, \cspdomain{Tank Allocation}, \cspdomain{Interview Assignment}, \cspdomain{Balanced Nursing Workload} and \cspdomain{Target Tracking} (\csplib problems $34$, $51$, $62$, $69$ and $72$).
%
%%%%%%%%%%%%%%%%%%%%%%%%%%%%%%%%%%%%%%%%%%%%%%%%%%%%%%%%%%%%%%%%%%%%%%%%%%%%%%%%
\subsection{Multi-Vertex Symmetries}
%%%%%%%%%%%%%%%%%%%%%%%%%%%%%%%%%%%%%%%%%%%%%%%%%%%%%%%%%%%%%%%%%%%%%%%%%%%%%%%%
%
As vertices in graph models are associated with both a row and column in the adjacency matrix, interchangeable vertices result in special cases of multi-row-column symmetry, termed here \emph{vertex symmetries}, in which rows $i$ and $j$ are swapped \IFF columns $i$ and $j$ are also:
%
\begin{defn}\label{def:vertex-symm} Let $\mat{M}$ be an $n$-vertex graph model with symmetry $\perm$. 
Then, $\perm$ is a \defterm{vertex symmetry} \IFF there exists an $i,j$ \ST $\perm = \rowperm{i}{j} \permcomp \colperm{i}{j}$.
\end{defn}
%
A \emph{multi-vertex symmetry} is a generalised vertex symmetry in which multiple pairs of vertices are simultaneously swapped:
%They are expressible as the composition of vertex symmetries $\perm_1 \permcomp \cdots \permcomp \perm_n$, where a row or column is swapped by at most one symmetry:
%
\begin{defn}\label{def:multi-vertex-symm} Let $\mat{M}$ be an $n$-vertex graph model with symmetry $\perm$. 
Then, $\perm$ is a \defterm{multi-vertex symmetry} \IFF there exist two disjoint sets of indices $i_1,\ldots,i_n$ and $j_1,\ldots,j_n$ \ST $\perm = \rowperm{i_1}{j_1} \permcomp \cdots \permcomp \rowperm{i_n}{j_n} \permcomp \colperm{i_1}{j_1} \permcomp \cdots \permcomp \colperm{i_n}{j_n}$.  
\end{defn}
%
\subimport*{graphics/}{hamilton-example.tex}
%
For example, Figure~\ref{fig:span-example} depicts two solutions to an instance of the \cspdomain{Hamiltonian Path} problem, which asks whether a graph has a path that visits each vertex exactly once.
As swapping $\gvert_1$ and $\gvert_2$ with $\gvert_3$ and $\gvert_4$, respectively, leaves $\mat{G}$ unchanged, this instance has a multi-vertex symmetry $\perm = \rowperm{1}{3} \permcomp \rowperm{2}{4} \permcomp \colperm{1}{3} \permcomp \colperm{2}{4}$ that swaps rows and columns one and two with rows and columns three and four. 
Path $\mat{P}_2$ is the image of $\mat{P}_1$ under $\perm$.
To make this transformation clearer, adjacency matrix $\mat{P}'$ in Figure~\ref{fig:span-example-adj-int} depicts an ``intermediate'' state, in which rows one and two have been swapped with rows three and four (as shaded).
Swapping the columns in $\mat{P}'$ (as outlined) completes the transformation into $\mat{P}_2$.

More complex graph models with multi-vertex symmetry include \cspdomain{Synchronous Optical Networking}, \cspdomain{Diameter and Degree Bounded Network Design Problem} and \cspdomain{Transshipment} (\csplib problems $56$, $71$ and $83$).
%
%%%%%%%%%%%%%%%%%%%%%%%%%%%%%%%%%%%%%%%%%%%%%%%%%%%%%%%%%%%%%%%%%%%%%%%%%%%%%%%%
\subsection{Generalised Symmetry Groups}\label{sec:gen-symm-grps}
%%%%%%%%%%%%%%%%%%%%%%%%%%%%%%%%%%%%%%%%%%%%%%%%%%%%%%%%%%%%%%%%%%%%%%%%%%%%%%%%
%
In the examples above, multi-row-column symmetries occur when a matrix or graph model contains interchangeable combinations of domain objects.
The resulting symmetries that swap the rows and/or columns associated with a pair of such combinations can be closed under composition (Definition~\ref{def:group}) to form the symmetric group (Definition~\ref{def:symmetric-grp}) over the combinations.
If there are $n$ combinations, this group can (as with all symmetric groups) be generated by $n-1$ symmetries, where each $\perm_i$ swaps all rows and columns in combination $i$ with their counterparts in combination $i+1$.

\subimport*{graphics/}{tank-example.tex}
%
For example, Figure~\ref{fig:tank-example} (ignoring highlighted cells for now) shows two symmetrical solutions to a simple instance of the \cspdomain{Tank Allocation} problem (\csplib problem~$51$), in which an assignment of chemicals to tanks must be found that satisfies adjacency and tolerance constraints.
The instance comprises six tanks $t_1,\ldots,t_6$, and four chemicals $c_1,\ldots,c_4$.
The tank configuration is in Figure~\ref{fig:tank-config}, with tanks annotated with their tolerances, and chemicals $c_1$--$c_3$ cannot be stored in adjacent tanks.

This instance contains three combinations of interchangeable objects: $\tup{t_1, t_2, c_1}$, $\tup{t_3, t_4, c_2}$, and $\tup{t_5, t_6, c_3}$.
Swapping the elements of any one with their counterparts in any other leaves Figure~\ref{fig:tank-example} unchanged.
This instance's symmetry group $\group$ is thus, essentially, the symmetric group over the three combinations of objects, and can be generated from $\generator = \set{\perm_1, \perm_2}$, where $\perm_1 = \rowperm{1}{3} \permcomp \rowperm{2}{4} \permcomp \colperm{1}{2}$ swaps the first and second combinations, and $\perm_2 = \rowperm{3}{5} \permcomp \rowperm{4}{6} \permcomp \colperm{2}{3}$ swaps the second and third.
%As $\group$ is generated by a set of multi-row-column symmetries, it is termed a \emph{multi-row-column symmetry group}.

More formally, symmetry groups of this type are termed \emph{multi-row-column symmetry groups}, and occur when a subset of a matrix or graph model's rows and/or columns can be partitioned into equal-sized, interchangeable combinations:
%
\begin{defn}\label{def:multi-row-col-symm-grp} Let $\mat{M}$ be a matrix or graph model with symmetry group $\group$. 
Then, $\group$ is a \defterm{multi-row-column symmetry group} over $\mat{M}$ \IFF there exist $n$ equal-length, disjoint lists of row indices $\vec{r}_1,\ldots,\vec{r}_n$, and $n$ equal-length, disjoint lists of column indices $\vec{c}_1,\ldots,\vec{c}_n$, and $\group$ has a generator $\generator = \set{\perm_1,\ldots,\perm_{n-1}}$ where for $1 \leq i < n$, $\perm_i$ is a multi-row-column symmetry as follows:
\begin{align*}
    \perm_i = 
    & \permcomp \set{\rowperm{j}{k} : j = \vec{r}_i[n], k = \vec{r}_{i+1}[n], 1 \leq n \leq \card{\vec{r}_i}} \permcomp \\
        & \permcomp \set{\colperm{j}{k} : j = \vec{c}_i[n], k = \vec{c}_{i+1}[n], 1 \leq n \leq \card{\vec{c}_i}}.
\end{align*}
\end{defn}
%
When the result of $n$ interchangeable combinations of rows and/or columns, multi-row-column symmetry groups are isomorphic to $S_n$ and are thus of size $n!$.

Special cases arise when $\generator$ contains more specific symmetry types. 
For example, when $\generator$ comprises multi-row symmetries (i.e., all $\card{\vec{c}_i} = 0$), $\group$ is termed a \emph{multi-row symmetry group}, and when $\mat{M}$ is a graph model and $\generator$ comprises multi-vertex symmetries (i.e., all $\vec{r}_i = \vec{c}_i$), $\group$ is a termed a \emph{multi-vertex symmetry group}, respectively.

A natural addition to this taxonomy are \emph{multi-row-and-multi-column symmetry groups}, in which multiple rows and columns are swapped, but rows and columns need not be swapped simultaneously (as in the \cspdomain{Vessel Loading} example in Figure~\ref{fig:vessel-example}).
Such groups are expressible as the direct product (Definition~\ref{def:direct-product}) $\group = \group_r \times \group_c$ of a multi-row symmetry group $\group_r$ and a multi-column symmetry group $\group_c$. 
%
% comparison to partitioned symmetries (ie codish et al) aka piecewise symmetries (Hentenryck -- Compositional Derivation of Symmetries for Constraint Satisfaction)
%
%%%%%%%%%%%%%%%%%%%%%%%%%%%%%%%%%%%%%%%%%%%%%%%%%%%%%%%%%%%%%%%%%%%%%%%%%%%%%%%%
\subsection{Symmetry Hierarchy}\label{sec:gen-symm-hierarchy}
%%%%%%%%%%%%%%%%%%%%%%%%%%%%%%%%%%%%%%%%%%%%%%%%%%%%%%%%%%%%%%%%%%%%%%%%%%%%%%%%
%
\subimport*{graphics/}{symm-hierarchy.tex}
%
Figure~\ref{fig:symm-hierarchy} depicts the hierarchy of the symmetry groups presented above, with arrows leading from less to more general groups.
Those shaded in grey have been previously studied (Sections~\ref{sec:symm-break-matrix} and~\ref{sec:symm-break-graph}), and the remainder are defined above.
%Those shaded in grey have been well studied, and can be (partially) broken with existing approaches, such as \dblex or \snakelex for matrix models, or the approaches studied by \citet{Codish2013:SymmGraphReps} and \citet{Shlyakhter2007:SymmBreaking} for graph models.
%In many cases, the difference between row and column symmetries and their more general forms is that latter permute many rows and/or columns simultaneously.
Generalisations from single to multiple interchangeable rows, columns, or vertices are marked with dotted lines.

The more general symmetry groups can be considered ``weaker'' in that they result in fewer symmetrical solutions.
Intuitively, this is because more general symmetries are ``implied'' by less general ones: if all rows are interchangeable, then any equal-sized combinations will be also, but of course the opposite does not hold.

Less intuitively, as generalised symmetries are the composition of row and/or column symmetries (Definitions~\ref{def:row-col-symm}~and~\ref{def:row-col-symm}), and groups are closed under composition, any symmetry group resulting from interchangeable combinations of rows and/or columns must be a subgroup (Definition~\ref{def:subgroup}) of the one created when those same rows and columns are individually interchangeable.
Since two solutions are symmetrical \WRT to a symmetry group \IFF one is the image of the other under an element of the group, if $\group_1 \leq \group_2$, then $\group_1$ cannot introduce more symmetry than $\group_2$: any two solutions symmetrical under $\group_1$ must also be symmetrical under $\group_2$, but the opposite need not hold.
