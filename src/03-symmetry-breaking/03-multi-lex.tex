%%%%%%%%%%%%%%%%%%%%%%%%%%%%%%%%%%%%%%%%%%%%%%%%%%%%%%%%%%%%%%%%%%%%%%%%%%%%%%%%
\section{Breaking Generalised Symmetries with \multilex}\label{sec:multi-lex}
%%%%%%%%%%%%%%%%%%%%%%%%%%%%%%%%%%%%%%%%%%%%%%%%%%%%%%%%%%%%%%%%%%%%%%%%%%%%%%%%
%
Standard approaches to symmetry breaking in matrix and graph models are too strong for the generalised symmetries introduced above.
Thus, this section introduces \multilex, an extension of \dblex that breaks multi-row-column symmetries (and by extension, multi-row, multi-column, and multi-vertex symmetries) by extending the model with lexicographic constraints (Definition~\ref{def:lex-order}).
%
\subsection{Multi-Row-Column Symmetry Example}
%
For example, the \cspdomain{Tank Allocation} example in Figure~\ref{fig:tank-example}, has a multi-row-column symmetry group generated by $\perm_1 = \rowperm{1}{3} \permcomp \rowperm{2}{4} \permcomp \colperm{1}{2}$ and $\perm_2 = \rowperm{3}{5} \permcomp \rowperm{4}{6} \permcomp \colperm{2}{3}$.
Solution $\mat{A}_1$ is in canonical form, however, attempting to break $\perm_2$ with \dblex by requiring that, for example, $\mat{A}[3] \lexleq \mat{A}[5]$, $\mat{A}[4] \lexleq \mat{A}[6]$ and $\mat{A}[][2] \lexleq \mat{A}[][3]$\footnote{As $\mat{A}$ is a 0/1 Boolean matrix, lexicographic comparison amounts to comparison of binary numbers.}, would rule out both $\mat{A}_1$ and $\mat{A}_2$. 
Indeed, \dblex would rule out all solutions from this symmetry class, indicating that it is \emph{too strong to apply to this form of symmetry}.

However, \dblex can be generalised to break multi-row-column symmetries.
To explain the generalisation, it is helpful to consider an alternative definition of \dblex.
Under the standard definition, if $\mat{M}$ is a matrix with row symmetry $\rowperm{i}{j}$ where $i < j$, \dblex requires that $\mat{M}[i] \lexleq \mat{M}[j]$.
Alternatively, the left-hand side of the constraint, $\mat{M}[i]$, can be defined as the list of all variables in $\mat{M}$, in their original order, that are mapped into a higher row by $\rowperm{i}{j}$. 
As $\rowperm{i}{j}(\mat{M}[i]) = \mat{M}[j]$, the right-hand side is simply the image of the left.
\multilex breaks multi-row-column symmetries with lex-leader constraints constructed in the same way.
Each constraint is of the form $\vec{v} \lexleq \perm(\vec{v})$, where $\vec{v}$ contains all variables that are mapped into a higher row or column by $\perm$, ordered as they would be if the matrix's rows were placed end-to-end.

Returning to Figure~\ref{fig:tank-example}, as $\perm_2$ above swaps rows $3$ and $5$, rows $4$ and $6$ and columns $2$ and $3$, $\vec{v}$ in this case contains the variables in the matrix's third and fourth rows, and second column (as highlighted in Figure~\ref{fig:tank-sol1}).
The variables in $\vec{v}$ are ordered as they would be if the matrix's rows were placed end to end, that is, $\vec{v}$ contains the second element of the first and second rows, then the entire third and fourth rows, and then the second element of the fifth and sixth rows.
When breaking $\perm_2$ in $\mat{A}_1$, \multilex compares the highlighted cells in Figure~\ref{fig:tank-sol1} with the highlighted cells in Figure~\ref{fig:tank-sol2} (since $\perm_2(\mat{A}_1) = \mat{A}_2$).
As $000100000000 \lexless 000100010000$, \multilex holds for $\mat{A}_1$, but not $\mat{A}_2$.

Due to the factorial size of multi-row-column symmetry groups, \multilex only explicitly breaks the symmetries found in the symmetry group's generating set.
Thus, like \dblex, \multilex trades tractability for completeness, meaning that while it holds for canonical solutions (i.e., it is \emph{correct}), there may be non-canonical solutions for which it also holds.
%
\subsection{\DAG with Self-Loops Example}
%
\subimport*{graphics/}{dag-sl-example.tex}
%
Unlike the standard graph symmetry breaking techniques discussed in Section~\ref{sec:symm-break-graph}, the approach described above can also break symmetries in directed graphs with self-loops.
Graphs $\mat{G}_1$ and $\mat{G}_2$ in Figure~\ref{fig:dagsl-example} are solutions to a three-vertex \DAG model with the constraint that vertices $\gvert_1$ and $\gvert_2$ have an out-degree of at least one.
As $\gvert_1$ and $\gvert_2$ are interchangeable, $\mat{G}_1$ and $\mat{G}_2$ are symmetrical under the vertex symmetry $\perm = \rowperm{1}{2} \permcomp \colperm{1}{2}$.

\multilex breaks this symmetry with a constraint of the form $\vec{v} \lexleq \perm(\vec{v})$, where $\vec{v}$ contains all variables in the first row, and then the first elements of the second and third rows.
When applied to $\mat{G}_1$, \multilex compares the highlighted cells in Figure~\ref{fig:dagsl-example-1} with the highlighted cells in Figure~\ref{fig:dagsl-example-2} (since $\perm(\mat{G}_1) = \mat{G}_2$).
As $00100 \lexless 10000$, \multilex holds for $\mat{G}_1$, but not $\mat{G}_2$.
\multilex will now be defined more formally.
%
%%%%%%%%%%%%%%%%%%%%%%%%%%%%%%%%%%%%%%%%%%%%%%%%%%%%%%%%%%%%%%%%%%%%%%%%%%%%%%%%
\subsection{Canonical Matrices}
%%%%%%%%%%%%%%%%%%%%%%%%%%%%%%%%%%%%%%%%%%%%%%%%%%%%%%%%%%%%%%%%%%%%%%%%%%%%%%%%
%
As \multilex aims to remove as many non-canonical solutions as possible, canonicity in this context must be defined.
As is standard, the lexicographic least solution in each symmetry class is selected as the canonical form, with the lexicographic ordering deriving from a \emph{row-wise ordering} of variables:
%
\begin{defn} The \defterm{row-wise ordering} of a matrix $\mat{M}$ is the vector $\mat{M}_R = \mat{M}[1] \concat \cdots \concat \mat{M}[n]$.
\end{defn}
%
A solution is a substitution $\sub$ (Section~\ref{sec:notation}) that maps variables to constants, and is \emph{canonical} \IFF the resulting matrix is lexicographically less than or equal to its image under all symmetries:
%
\begin{defn}\label{def:can-matrix} If $\mat{M}$ is a matrix of variables with symmetry group $\group$, and solution $\sub$ is a ground substitution that is complete \WRT to the variables in $\mat{M}$, then:
%
\begin{equation*}
\canonical(\mat{M}, \group, \sub) \eqdef \bigwedge_{\perm \in \group} \sub(\mat{M}_R) \lexleq \sub(\perm(\mat{M}_R)).
\end{equation*}
\end{defn}
%
The canonicity definition above compares lists of constants. 
On the left-hand side, the variables in $\mat{M}$ have been placed in a row-wise ordering and then replaced by their bindings under $\sub$, and on the right-hand side the variables have been permuted by $\perm$ before being ordered and replaced.
%
%%%%%%%%%%%%%%%%%%%%%%%%%%%%%%%%%%%%%%%%%%%%%%%%%%%%%%%%%%%%%%%%%%%%%%%%%%%%%%%%
\subsection{Breaking Multi-Row-Column Symmetries}
%%%%%%%%%%%%%%%%%%%%%%%%%%%%%%%%%%%%%%%%%%%%%%%%%%%%%%%%%%%%%%%%%%%%%%%%%%%%%%%%
%
\multilex posts a lex-leader constraint for each element of the symmetry group's generating set.
For each generator $\perm$, a vector is constructed by limiting the matrix's row-wise ordering to variables in a row or column that $\perm$ maps to a higher row or column. 
The lex-leader requires that this vector be lexicographically less than or equal to its image under $\perm$.

If $\perm$ is a multi-row-column symmetry, then the sets $R_\perm$ and $C_\perm$ contain the indices of rows or columns, respectively, that are mapped to a higher row or column by $\perm$:
%
\begin{defn} Let $\mat{M}$ be a matrix with multi-row-column symmetry $\perm = \perm_r \permcomp \perm_c$, where $\perm_r$ and $\perm_c$ are a multi-row and a multi-column symmetry, respectively. 
Then, $R_\perm = \set{i: \perm_r(\mat{M}[i]) = \mat{M}[j], i < j}$, and $C_\perm = \set{i: \perm_c(\mat{M}[][i]) = \mat{M}[][j], i < j}$.
\end{defn}
%
For example, if $\perm = \rowperm{1}{4} \permcomp \rowperm{3}{5} \permcomp \colperm{4}{6}$, that is, rows $1$ and $3$ column $4$ are swapped with rows $4$ and $5$ and column $6$, then $R_\perm = \set{1, 3}$ and $C_\perm = \set{4}$.

If $\mat{M}$ is a matrix and $R, C$ are sets of integers, then $\mat{M} \matfilter \tup{R, C}$ denotes the restriction of the row-wise ordering of $\mat{M}$ to variables that have either a row index in $R$ or column index in $C$:
%
\begin{defn} If $\mat{M}$ is an $r \times c$ matrix, $R \subseteq \set{1,\ldots,r}$ and $C \subseteq \set{1,\ldots,c}$ are sets of integers, and $\vec{c}$ is the natural total order over $C$ of length $n$, then $\mat{M} \matfilter \tup{R, C}$ denotes the vector $\vec{x_1} \concat \cdots \concat \vec{x_r}$, where for $1 \leq i \leq r$:
\begin{equation*}
    \vec{x}_i = \begin{cases}
        \mat{M}[i] & \text{if } i \in R \\
        \tup{\mat{M}[i][\vec{c}[1]],\ldots,\mat{M}[i][\vec{c}[n]]} & \text{if } i \not\in R
    \end{cases}
\end{equation*}
\end{defn}
%
For example, if $\mat{M} = \tup{\tup{x_1, x_2, x_3}, \tup{x_4, x_5, x_6}, \tup{x_7, x_8, x_9}}$, then $\mat{M} \vecfilter \tup{\set{1, 2}, \set{2}}$ contains first and second rows of $\mat{M}$, and the second element of the third row: $\tup{x_1, x_2, x_3,x_4, x_5, x_6, x_8}$.

Using the above definitions, \multilex is defined as follows:
%	
\begin{defn}\label{def:multilex} If $\mat{M}$ is a matrix with multi-row-column symmetry group $\group$, and $\generator$ is a generating set of $\group$ as in Definition~\ref{def:multi-row-col-symm-grp}, then the \defterm{Multi Lex} constraint for $\mat{M}$ and $\generator$ is defined as follows:
\begin{align*}
\MultiLex(\mat{M}, \generator) \eqdef \bigwedge_{\perm \in \generator} \mat{M} \matfilter \tup{R_\perm, C_\perm} \lexleq \perm(\mat{M}) \matfilter \tup{R_\perm, C_\perm}.
\end{align*}	
\end{defn}	
% 
A constraint is posted for each element of the generating set.
On the left-hand side, the variables in $\mat{M}$ have been placed in row-wise order, and then limited to those that are mapped to a higher row or column under $\perm$ (i.e., with a row or column number in $R_\perm$ or $C_\perm$). 
On right-hand side they have been permuted by $\perm$ before being ordered and limited.

The following theorem establishes that \multilex holds for canonical matrices, meaning that it is a correct symmetry breaking constraint:\footnote{Proofs for all theorems in this chapter are in Appendix~\ref{apx:symmetry-breaking}.}
%
\begin{restatable}{theorem}{mlex}\label{thrm:mlex}
\csprob{Multi Lex Correctness}.
Let $\mat{M}$ be a matrix with multi-row-column symmetry group $\group$, $\generator$ be a generating set of $\group$ as in Definition~\ref{def:multi-row-col-symm-grp} and $\sub$ be a ground substitution that is complete \WRT to the variables in $\mat{M}$.
If $\canonical(\mat{M}, \group, \sub)$ and $\perm \in \generator$, then $\sub(\mat{M}) \matfilter \tup{R_\perm, C_\perm}) \lexleq \perm(\sub(\mat{M})) \matfilter \tup{R_\perm, C_\perm}$ holds.
\end{restatable}
%
\multilex posts a polynomial-sized set of constraints, that in the worst case is of size $4rc$:
%
\begin{restatable}{theorem}{mlexsize}\label{thrm:mlex-size-comp}
\csprob{Multi Lex Size Complexity}.
Let $\mat{M}$ be an $r \times c$ matrix with multi-row-column symmetry group $\group$, and $\generator$ be the generating set of $\group$ as in Definition~\ref{def:multi-row-col-symm-grp}.
Then, \multilex posts constraints of size $\bigo(rc)$.
\end{restatable}
%
\paragraph{Comparison with \dblex} It is clear from the above that when applied to row or column symmetry groups, \multilex reduces to \dblex.
For example, if all rows are interchangeable, then from Definition~\ref{def:multi-row-col-symm-grp}, $\generator = \set{\perm_1,\ldots,\perm_{r-1}}$ where each $\perm_i$ swaps rows $i$ and $i+1$.
In this case, $\mat{M} \matfilter \tup{R_{\perm_i}, C_{\perm_i}}$ resolves to $\mat{M}[i]$, meaning that \multilex, like \dblex, requires the rows to lexicographically ordered, that is, for $1 \leq i \leq r-1$, $\mat{M}[i] \lexleq \mat{M}[i+1]$.

However, unlike \dblex, \multilex can be applied to a broader range of symmetries, including those found in arbitrary graph models (i.e., directed with cycles and self-loops allowed).
This flexibility comes with no additional size cost, as both methods produce constraints smaller than $4rc$.