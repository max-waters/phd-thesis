\newcommand{\spangraph}[1]{
\begin{tikzpicture}[
    sensor/.style={draw,circle,text=black,font=\small,inner sep=0pt,text width=4mm,align=center},
    target/.style={draw,circle,text=black,font=\scriptsize,inner sep=0pt,text width=3mm,align=center,fill=gray!50},
]
    
    \node[sensor] (s1) []	{$\gvert_1$};
    \node[sensor] (s2) [right = 1cm of s1.center, anchor=center]	{$\gvert_2$};  
    \node[sensor] (s3) [below = 1cm of s1.center, anchor=center]	{$\gvert_3$};
    \node[sensor] (s4) [right = 1cm of s3.center, anchor=center]	{$\gvert_4$};
    \node[sensor] (s5) [below right = 0.5cm and 1cm of s2.center, anchor=center]	{$\gvert_5$};

    \draw (s1) -- (s3);
    \draw (s2) -- (s5) -- (s4);
    #1
    
\end{tikzpicture}%
}

\newcommand{\spanmatrix}[5]{
\begin{tabular}{ *{5}{p{0.1cm}} }
#1 \\
#2 \\
#3 \\
#4 \\
#5 \\
\end{tabular}
}

\newcommand{\tikzmark}[1]{\tikz[overlay,remember picture] \node (#1) {};}

\newcommand{\DrawBox}[3][]{%
    \tikz[overlay,remember picture]{
    \draw[black,#1]
      ($(#2)+(-0.75em,1.75ex)$) rectangle
      ($(#3)+(0.75em,-0.5ex)$);}
}

\begin{figure}[t]
    \begin{subfigure}[t]{0.33\textwidth}
        \centering
        \spangraph{\draw (s2) -- (s4) -- (s3) -- (s2) -- (s1) -- (s4);}
        \caption{Graph $\mat{G}$.}
        \label{fig:span-example-graph}
    \end{subfigure}%
    \begin{subfigure}[t]{0.33\textwidth}
        \centering
        \spangraph{\draw (s3) -- (s4);}
        \caption{Hamiltonian path $\mat{P}_1$.}
        \label{fig:span-example-tree1}
    \end{subfigure}% 
    \begin{subfigure}[t]{0.33\textwidth}
        \centering
        \spangraph{\draw (s1) -- (s2);}
        \caption{Hamiltonian path $\mat{P}_2$.}
        \label{fig:span-example-tree2}
    \end{subfigure}%

    \vspace{0.5cm}

    \begin{subfigure}[t]{0.33\textwidth}
        \centering
        \spanmatrix{  \rowcolor{lightgray} 0 & 0 & 1 & 0 & 0}%
                    { \rowcolor{lightgray} 0 & 0 & 0 & 0 & 1}%
                    { \rowcolor{gray!20}1 & 0 & 0 & 1 & 0}%
                    { \rowcolor{gray!20}0 & 0 & 1 & 0 & 1}%
                    { 0 & 1 & 0 & 1 & 0}%

        \caption{Adjacency matrix of path $\mat{P}_1$.}
        \label{fig:span-example-adj1}
    \end{subfigure}% 
    \begin{subfigure}[t]{0.33\textwidth}
        \centering
        \spanmatrix{  \rowcolor{gray!20} 1\tikzmark{top left 1} & 0 & 0\tikzmark{top left 2} & 1 & 0}%
                    { \rowcolor{gray!20} 0 & 0 & 1 & 0 & 1}%
                    { \rowcolor{lightgray} 0 & 0 & 1 & 0 & 0}%
                    { \rowcolor{lightgray} 0 & \tikzmark{bottom right 1}0 & 0 & \tikzmark{bottom right 2}0 & 1}%            
                    { 0 & 1 & 0 & 1 & 0}%
        \DrawBox[]{top left 1}{bottom right 1}
        \DrawBox[dashed]{top left 2}{bottom right 2}
        \caption{Intermediate matrix $\mat{P}'$.}
        \label{fig:span-example-adj-int}
    \end{subfigure}% 
    \begin{subfigure}[t]{0.33\textwidth}
        \centering
        \spanmatrix{ 0\tikzmark{top left 3} & 1 & 1\tikzmark{top left 4} & 0 & 0}%
                    {1 & 0 &  0 &  0 & 1}%
                    { 1 & 0 & 0 & 0 & 0}%
                    { 0 & \tikzmark{bottom right 3}0 & 0 & \tikzmark{bottom right 4}0 & 1}%
                    {0 & 1 & 0 & 1 & 0}%
        \DrawBox[dashed]{top left 3}{bottom right 3}
        \DrawBox[]{top left 4}{bottom right 4}
        \caption{Adjacency matrix of path $\mat{P}_2$.}
        \label{fig:span-example-adj2}
    \end{subfigure}%
\caption[\cspdomain{Hamiltonian Path} example]{Two solutions to a \cspdomain{Hamiltonian Path} problem.}
\label{fig:span-example}
\end{figure}