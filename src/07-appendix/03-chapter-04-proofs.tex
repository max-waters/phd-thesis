\chapter{Proofs of Theorems in Chapter~\ref{chap:pop-maxsat}}\label{apx:pop-maxsat}

%%%%%%%%%%%%%%%%%%%%%%%%%%%%%%%%%%%%%%%%%%%%%%%%%%%%%%%%%%%%%%%%%%%%%%%%%%%%%%%%
\section{Proof of Theorem~\ref{thrm:mrd-mrr-npc}}
%%%%%%%%%%%%%%%%%%%%%%%%%%%%%%%%%%%%%%%%%%%%%%%%%%%%%%%%%%%%%%%%%%%%%%%%%%%%%%%%

Theorem~\ref{thrm:mrd-mrr-npc} states that determining whether a \POP has a minimum reinstantiated deordering or reordering with less than $k$ ordering constraints are both \NP-complete.
A proof of the \NP-completeness of minimum reinstantiated reordering is presented below.
The proof for minimum reinstantiated deordering is a trivial modification.

\mmrd*

\begin{proof} Let $P = \tup{\opset, \sub, \precrel}$ be a \POP and $k > 0$ an integer. 
To prove membership in \NP, guess a \POP $Q = \tup{\opset, \sub', \precrel'}$ and verify in polynomial time that $Q$ is valid, and $\card{\precrel'} < k$.

Proof of \NP-hardness is by reduction from the decision problem \MINREORDER, which is \NP-complete~\cite{Backstrom-CompAspects} and asks whether, for any \POP $P = \tup{\opset, \sub, \precrel}$ and integer $k > 0$, there exists a valid \POP $Q = \tup{\opset, \sub, \precrel'}$ such that $\card{\precrel} < k$ (Definition \ref{def:opt-de-reorder}).
Let $P = \tup{\opset, \sub, \precrel}$ be a \POP such that $\opset = \set{\initoptr, \optr_1,\ldots,\optr_n,\goaloptr}$.
First construct, from each constant $c \in \consts(\opset)$, the predicate symbol $is_c$.
Next, construct the \POP $P' = \tup{\opset', \sub, \precrel}$ such that $\opset' = \set{\initoptr', \optr_1',\ldots,\optr_n',\goaloptr}$, where $\pre(\initoptr') = \emptyset$, $\post(\initoptr') = \post(\initoptr) \cup \set{is_c(c) : c \in \consts(\opset)}$, and for $1 \leq i \leq n$, $\vars(\optr_i') = \vars(\optr_i)$, $\post(\optr_i') = \post(\optr_i)$ and $\pre(\optr_i') = \pre(\optr_i) \cup \set{is_c(x) : x \in \vars(\optr_i'), c = \sub(x)}$.
The $is_c$ preconditions in $P'$'s operators mean that the bindings of all variables in $P'$ are ``fixed'': any valid reinstantiation of $P'$ cannot alter $\sub$.
It follows that there exists a reorder of $P$ with fewer than $k$ ordering constraints \IFF there exists a reinstantiated reorder of $P'$ with fewer than $k$ ordering constraints.
\end{proof}


%%%%%%%%%%%%%%%%%%%%%%%%%%%%%%%%%%%%%%%%%%%%%%%%%%%%%%%%%%%%%%%%%%%%%%%%%%%%%%%%
\section{Proof of Theorem~\ref{thrm:mrd-mrr-opt}}
%%%%%%%%%%%%%%%%%%%%%%%%%%%%%%%%%%%%%%%%%%%%%%%%%%%%%%%%%%%%%%%%%%%%%%%%%%%%%%%%

Theorem~\ref{thrm:mrd-mrr-opt} states that finding a minimal reinstantiated de/reordering of a \POP cannot be approximated to within a constant factor.
As above, a proof for minimum reinstantiated reordering is presented.
The proof for minimum reinstantiated deordering is a trivial modification.

\mmrdopt*

\begin{proof} As the polynomial time reduction in the proof of Theorem~\ref{thrm:mrd-mrr-npc} also preserves solutions, it is therefore an approximation preserving reduction. 
As \MINREORDER cannot be approximated within a constant factor unless $\NP \subseteq \DTIME(n^{\text{poly log n}})$~\cite{Backstrom-CompAspects}, it follows that a minimum reinstantiated reorder cannot be either.
\end{proof}


\section{Proof of Theorem~\ref{thrm:mrr-encoding}}

Theorem~\ref{thrm:mrr-encoding} asserts that an optimal solution to the \MRR/\MRD \MAXSAT encoding of a \POP $P$ corresponds to a minimum reinstantiated de/reorder of $P$, respectively.
This will be proved for the case of \MRR, the case for \MRD is a minor modification.

The following lemma establishes that any \POP corresponding to a solution to the \MRR \MAXSAT encoding as in Definition~\ref{def:mrr-soln} has well-formed ordering constraints:

%%%%%%%%%% acyclic, transitively closed
\begin{lemma}\label{lemma:mrr-encoding-wf-1} Let $P = \tup{\opset, \sub, \precrel}$ be a \POP, $S$ be a solution to the \MRR \MAXSAT encoding of $P$, and $\popfunc(S) = \tup{\opset, \sub', \precrel'}$ be the \POP corresponding to $S$.
Then, $\precrel'$ is acyclic and transitively closed.
\end{lemma}

\begin{proof} Assume that there is an $\optr_1$, $\optr_2$ and $\optr_3$ \ST $\optr_1 \prec' \optr_2$ and $\optr_2 \prec' \optr_3$.
From Definition~\ref{def:mrr-soln}, $\precp{\optr_1}{\optr_2}, \precp{\optr_2}{\optr_3} \in \precpset$ and $S(\precp{\optr_1}{\optr_2}) = S(\precp{\optr_2}{\optr_3}) = \true$.
Therefore, $\precp{\optr_1}{\optr_3} \in \precpset$ (Definition~\ref{def:maxsat-props-sets}), and from Formula~\ref{eq:mrr-trans}, $S(\precp{\optr_1}{\optr_3}) = \true$, and so $\optr_1 \prec' \optr_3$.
Therefore, $\precrel'$ is transitively closed.

Assume that there is an $\optr_1$, $\optr_2$ \ST $\optr_1 \prec' \optr_2$.
From Definition~\ref{def:mrr-soln}, $\precp{\optr_1}{\optr_2} \in \precpset$, and $S(\precp{\optr_1}{\optr_2}) = \true$.
Either $\precp{\optr_2}{\optr_1} \in \precpset$, or not.
Assume that $\precp{\optr_2}{\optr_1} \in \precpset$. 
Then, from Formula~\ref{eq:mrr-asymm}, $S(\precp{\optr_2}{\optr_1}) = \false$, and so $\optr_2 \not\precrel' \optr_1$, and thus $\precrel'$ is acyclic.
Assume that $\precp{\optr_2}{\optr_1} \not\in \precpset$.
Then, from Definition~\ref{def:mrr-soln}, $\optr_2 \not\precrel' \optr_1$ and so $\precrel'$ is acyclic.
\end{proof}

%%%%%%% bindings lemma
The following lemma establishes that if variable $x$ is bound to constant $c$ in a \POCL-valid reinstantiated reorder of a \POP, then $\eqp{x}{c} \in \eqpset$:

\begin{lemma}\label{lemma:var-bind-eqp} Let $P = \tup{\opset, \sub, \precrel}$ be a \POCL-valid \POP with \POCL-valid reinstantiated reorder $Q = \tup{\opset, \sub', \precrel'}$.
Then, for each $x \in \vars(\opset)$ and $c \in \consts(\opset)$ \ST $\sub'(x) = c$, $\eqp{x}{c} \in \eqpset$.
\end{lemma}

\begin{proof} This proof assumes that every variable appears in an operator precondition, and constants only appear in the initial state.
This results in no loss of generality as this can be achieved in any \POP through the addition of ``dummy'' preconditions.

For every $t \in \vars(\opset) \cup \consts(\opset)$, a chain of causally-connected terms $\vec{k}_t$ can be defined as follows.
If $t \in \consts(\opset)$, then $\vec{k}_t = \tup{t}$.
If $t \in \vars(\opset)$, then $\vec{k}_t = \tup{t} \concat \vec{k}_{t'}$\footnote{Where $\vec{x}\concat\vec{y}$ indicates the concatenation of $\vec{x}$ and $\vec{y}$.}, where $t$ and $t'$ are causally connected via an unthreatened causal link $\tup{\optr_p, q(\vec{u}), \optr_c, q_1(\vec{s})}$ \ST there is an $i$ where $\vec{u}[i] = t'$ and $\vec{s}[i] = t$.
As $Q$ is \POCL-valid, such a $t'$ must exist for every $t \in \vars(\opset)$, and $\sub'(t') = \sub'(t)$.

From Definition~\ref{def:maxsat-props-sets}, for each $t \in \vars(\opset)$ and $s \in \vec{k}_t$, $\eqp{t}{s} \in \eqpset$, and from the definition of $\vec{k}_t$, all elements must to the same constant, that is, $\sub'(t) = \sub'(s)$.
As the last element in all such chains must be some $c \in \consts(\initoptr)$, it follows that for all $x \in \vars(\opset)$ \ST $\sub'(x) = c$, $\eqp{x}{c} \in \eqpset$.
\end{proof}

The following lemma establishes that any \POP corresponding to a solution to the \MRR \MAXSAT encoding as in Definition~\ref{def:mrr-soln} has well-formed binding constraints:

%%%%%%%%%% ground and complete bindings
\begin{lemma}\label{lemma:mrr-encoding-wf-2} Let $P = \tup{\opset, \sub, \precrel}$ be a \POP, $S$ be a solution to the \MRR \MAXSAT encoding of $P$, and $\popfunc(S) = \tup{\opset, \sub', \precrel'}$ be the \POP corresponding to $S$.
Then, $\sub'$ is ground and complete \WRT $\vars(\opset)$.
\end{lemma}
    
\begin{proof} It suffices to prove that \myi if $x \in \vars(\opset)$ then there is a $c$ \ST $S(\eqp{x}{c}) = \true$, and \myii if $x \in \vars(\opset)$, $c_1,c_2 \in \consts(\opset)$ and $S(\eqp{x}{c_1}) = S(\eqp{x}{c_2}) = \true$, then $c_1 = c_2$.

First, \myi is proved.
Let $x \in \vars(\opset)$. 
From Formula~\ref{eq:mrr-eq-domain}, not all propositions $\eqp{x}{c} \in \eqpset$ \ST $c \in \consts(\opset)$ can be set to \false.
It suffices to show that there is at least one $c \in \consts(\opset)$ \ST $\eqp{x}{c} \in \eqpset$.
Let $c = \sub'(x)$.
Then, from Lemma~\ref{lemma:var-bind-eqp}, $\eqp{x}{c} \in \eqpset$.
Proof of \myii follows directly from Formula~\ref{eq:mrr-eq-domain}.
\end{proof}

%%%%%%%%%% POCL-valid
Lemma~\ref{lemma:mrr-encoding-valid} establishes that any \POP corresponding to a solution to the \MRR \MAXSAT encoding as in Definition~\ref{def:mrr-soln} is \POCL-valid:

\begin{lemma}\label{lemma:mrr-encoding-valid} Let $P = \tup{\opset, \sub, \precrel}$ be a \POP, $S$ be a solution to the \MRR \MAXSAT encoding of $P$, and $\popfunc(S) = \tup{\opset, \sub', \precrel'}$ be the \POP corresponding to $S$.
Then, $\popfunc(S)$ is \POCL-valid.
\end{lemma}

\begin{proof} Let $\optr_c$ and $q(\vec{s})$ be an operator and literal \ST $\consms(\optr_c, q(\vec{s}))$.
To prove that $\popfunc(S)$ is \POCL-valid it suffices to show that $\optr_c$ and $q(\vec{s})$ are supported by an unthreatened causal link.

Formula~\ref{eq:mrr-cl} states that not all $\pclp{\optr_p}{q(\vec{u})}{\optr_c}{q(\vec{s})} \in \pclpset$ can be set to \false.
It first must be shown that there is at least one such $\pclp{\optr_p}{q(\vec{u})}{\optr_c}{q(\vec{s})} \in \pclpset$.

As $P$ is \POCL-valid, each $\optr_c$ and $q(\vec{s})$ \ST $\consms(\optr_c, q(\vec{s}))$ are supported by an unthreatened causal link $\tup{\optr_p, q(\vec{u}), \optr_c, q(\vec{s})}$, and so from Definition~\ref{def:maxsat-props-sets}, $\pclp{\optr_p}{q(\vec{u})}{\optr_c}{q(\vec{s})} \in \pclpset$. 
Therefore, there is at least one $\pclp{\optr_p}{q(\vec{u})}{\optr_c}{q(\vec{s})} \in \pclpset$. 

Let $\pclp{\optr_p}{q(\vec{u})}{\optr_c}{q(\vec{s})} \in \pclpset$ and $S(\pclp{\optr_p}{q(\vec{u})}{\optr_c}{q(\vec{s})}) = \true$.
Then, from Definition~\ref{def:maxsat-props-sets} and Formula~\ref{eq:mrr-cl-holds}, $\precp{\optr_p}{\optr_c} \in \precpset$ and $S(\precp{\optr_p}{\optr_c}) = \true$, and for $1 \leq i \leq \card{\vec{s}}$, $\eqp{\vec{u}[i]}{\vec{s}[i]} \in \eqpset$ and $S(\eqp{\vec{u}[i]}{\vec{s}[i]}) = \true$. 
Therefore, $\optr_p \precrel' \optr_c$, $\sub'(\vec{u}) = \sub'(\vec{s})$, that is, $\optr_c$ and $q(\vec{s})$ are supported by a causal link.

Let $\optr_t$ and $q(\vec{v})$ be a operator and literal \ST $\thrts(\optr_t, q(\vec{v}))$. 
Then, from Definition~\ref{def:maxsat-props-sets}, $\precp{\optr_t}{\optr_p}, \precp{\optr_c}{\optr_t} \in \precpset$ and for $1 \leq i \leq \card{\vec{s}}$, $\eqp{\vec{p}[i]}{\vec{v}[i]} \in \eqpset$.
And, from Formula~\ref{eq:mrr-cl-holds}, either $S(\precp{\optr_t}{\optr_p}) = \true$, $S(\precp{\optr_c}{\optr_t}) = \true$, or there is some $1 \leq i \leq \card{\vec{s}}$ \ST $S(\eqp{\vec{s}[i]}{\vec{v}[i]}) = \false$.
Therefore, either $\optr_t \precrel' \optr_p$, $\optr_c \precrel' \optr_t$ or $\sub'(\vec{v}) \neq \sub'(\vec{s})$, that is, the causal link is unthreatened.
\end{proof}

%%%%%%%%%% POCL-valid POP -> solution
Lemma~\ref{lemma:pop-to-mrr-encoding} establishes that any \POCL-valid reinstantiated reorder $Q$ of a \POP $P$ corresponds to a solution to the \MRR \MAXSAT encoding of $P$.
The exact form of this solution is defined as follows:

\begin{defn}\label{def:pop-to-sol} Let $P = \tup{\opset, \sub, \precrel}$ be a \POP and $Q = \tup{\opset, \sub', \precrel'}$ be a reinstantiated reorder of $P$.
Then, $S_Q$ denotes the transformation of $Q$ into a solution to the \MRR \MAXSAT encoding of $P$ \ST:
\begin{itemize}
    \item for all $\precp{\optr_1}{\optr_2} \in \precpset$, $S_Q(\precp{\optr_1}{\optr_2}) = \true$ \IFF $\optr_1 \prec' \optr_2$,
    \item for all $\eqp{t}{u} \in \eqpset$, $S_Q(\eqp{t}{u}) = \true$ \IFF $\sub'(t) = \sub'(u)$, and
    \item for all $\pclp{\optr_p}{\vec{u}}{\optr_c}{\vec{s}} \in \pclpset$, $S_Q(\pclp{\optr_p}{\vec{u}}{\optr_c}{\vec{s}}) = \true$ \IFF precondition $q(\vec{s})$ of $\optr_c$ is supported by postcondition $q(\vec{u})$ of $\optr_p$ and the causal link $\tup{\optr_p, q(\vec{u}),$ $\optr_c, q(\vec{s})}$ is unthreatened in $Q$.
\end{itemize}
\end{defn}

\noindent
Lemma~\ref{lemma:pop-to-mrr-encoding} assumes that $Q$ is \emph{\POCL-minimal}, as defined below:

\begin{defn} A \POP $P = \tup{\opset, \sub, \precrel}$ is \defterm{\POCL-minimal} \IFF it is \POCL-valid, and for all $Q = \tup{\opset, \sub, \precrel'}$ \ST $\precrel' \subset \precrel$, $Q$ is not \POCL-valid.
\end{defn}

This assumption results in no loss of generality as any \POCL-valid \POP can be made \POCL-minimal by greedily removing ordering constraints.

\begin{lemma}\label{lemma:pop-to-mrr-encoding} Let $P = \tup{\opset, \sub, \precrel}$ be a \POP, and $Q = \tup{\opset, \sub', \precrel'}$ be a \POCL-minimal reinstantiated reorder of $P$.
Then $S_Q$ is a solution to the \MRR \MAXSAT encoding of $P$.
\end{lemma}

\begin{proof} As $Q$ is \POCL-minimal, it follows that every ordering constraint is required to ensure that a consumer is supported by a producer, a causal link is protected from threats, or that $\precrel'$ is transitively closed.
Therefore, from Definitions~\ref{def:mrr-causal-rel} and~\ref{def:maxsat-props-sets}, if $\optr_1 \precrel' \optr_2$ then $\precp{\optr_1}{\optr_2}\in \precpset$.
As $Q$ is a well-formed \POP, and $S_Q(\precp{\optr_1}{\optr_2}) = \true$ \IFF $\optr_1 \precrel' \optr_2$ (Definition~\ref{def:pop-to-sol}), $S$ satisfies Formulae~\ref{eq:mrr-asymm}--\ref{eq:mrr-ig}.

Formula~\ref{eq:mrr-eq-symm} requires that for all $\eqp{s}{u}, \eqp{s}{u} \in \eqpset$, $S_Q(\eqp{s}{u}) = S_Q(\eqp{u}{s})$, which follows directly from Definition~\ref{def:pop-to-sol}.
Formula~\ref{eq:mrr-eq-trans} requires that for all $\eqp{s}{u}, \eqp{u}{v}, \eqp{s}{v} \in \eqpset$, if $S_Q(\eqp{s}{u}) = S_Q(\eqp{u}{v}) = \true$ then $S_Q(\eqp{s}{v}) = \true$.
From Definition~\ref{def:pop-to-sol}, $S_Q(\eqp{s}{u}) = S_Q(\eqp{u}{v}) = \true$ \IFF $\sub'(s) = \sub'(u) = \sub'(v)$, which implies $S_Q(\eqp{s}{v}) = \true$, as desired.
Formula~\ref{eq:mrr-eq-domain} requires that for each variable $x \in \vars(\opset)$ there is exactly one constant $c \in \consts(\opset)$ \ST $\eqp{x}{c} \in \eqpset$ and $S_Q(\eqp{x}{c}) = \true$.
As $\sub'$ is ground, there is exactly one $c$ \ST $\sub'(x) = c$.
From Lemma~\ref{lemma:var-bind-eqp}, $\eqp{x}{c} \in \eqpset$, and from Definition~\ref{def:pop-to-sol}, $S_Q(\eqp{x}{c}) = \true$, as desired.

Let $q(\vec{s})$ be a precondition of $\optr_c \in \opset$.
As $Q$ is \POCL-valid, $q(\vec{s})$ is supported by a postcondition $q(\vec{u})$ of $\optr_p \in \opset$ and causal link $\tup{\optr_p, q(\vec{u}), \optr_c, q(\vec{s})}$ is unthreatened in $Q$.
From Definitions~\ref{def:maxsat-props-sets} and~\ref{def:pop-to-sol}, $\pclp{\optr_p}{\vec{u}}{\optr_c}{\vec{s}} \in \pclpset$ and $S_Q(\pclp{\optr_p}{\vec{u}}{\optr_c}{\vec{s}}) = \true$, thus satisfying Formula~\ref{eq:mrr-cl}.
From Definition~\ref{def:maxsat-props-sets}, $\precp{\optr_p}{\optr_c} \in \precpset$, and for $1 \leq i \leq \card{\vec{s}}$, $\eqp{\vec{s}[i]}{\vec{u}[i]} \in \eqpset$.
As $\optr_p \prec \optr_c$ and $\sub'(\vec{s}) = \sub'(\vec{u})$, then from Definition~\ref{def:pop-to-sol}, $S_Q(\precp{\optr_p}{\optr_c}) = \true$ and for $1 \leq i \leq \card{\vec{s}}$, $S_Q(\eqp{\vec{s}[i]}{\vec{u}[i]}) = \true$.
Therefore, the first line of Formula~\ref{eq:mrr-cl-holds} is satisfied.
Let postcondition $\neg q(\vec{v})$ of operator $\optr_t$ be a threat to the above link.
Then, from Definition~\ref{def:maxsat-props-sets}, $\precp{\optr_t}{\optr_p}, \precp{\optr_c}{\optr_t} \in \precpset$, and for $1 \leq i \leq \card{\vec{s}}$, $\eqp{\vec{v}[i]}{\vec{u}[i]} \in \eqpset$.
As either $\optr_t \prec \optr_p$, $\optr_c \prec \optr_t$ or $\sub'(\vec{s}) \neq \sub'(\vec{v})$, then from Definition~\ref{def:pop-to-sol}, either $S_Q(\precp{\optr_t}{\optr_p}) = \true$, $S_Q(\precp{\optr_c}{\optr_t}) = \true$, or there is some $1 \leq i \leq \card{\vec{u}}$ \ST $S_Q(\eqp{\vec{u}[i]}{\vec{v}[i]}) = \false$. 
Thus, the second line of Formula~\ref{eq:mrr-cl-holds} is satisfied.
\end{proof}

\pagebreak
\noindent
Proof of Theorem~\ref{thrm:mrr-encoding} follows from the above lemmas:

\mrrenc*

\begin{proof} Let $Q = \popfunc(S)$. 
From Lemmas~\ref{lemma:mrr-encoding-wf-1}--\ref{lemma:mrr-encoding-valid}, $Q$ is a ``well-formed'' \POCL-valid \POP.
As $Q$ has the same operator set as $P$, then from Definition~\ref{def:reins-re-deorder}, $Q$ is a reinstantiated de/reorder of $P$.

Proof of minimality is by \emph{reductio}.
Let $R = \tup{\opset, \sub'', \precrel''}$ be a \POCL-valid \POP \ST $\precrel'' \subset \precrel'$.
From Lemma~\ref{lemma:pop-to-mrr-encoding}, there is some solution $S'$ \ST $\popfunc(S') = R$.
As $\precrel'' \subset \precrel'$, then from Definition~\ref{def:mrr-soln} $\set{\satprop : \satprop \in \precpset, S'(\satprop) = \true} \subset \set{\satprop : \satprop \in \precpset, S(\satprop) = \true}$.
As this contradicts the assumption that $S$ is optimal (Definition~\ref{def:mrr-soln}), either $R$ is not \POCL-valid, or $\precrel'' \not\subset \precrel'$.
Therefore, $Q$ is minimal.
\end{proof}


%%%%%%%%%%%%%%%%%%%%%%%%%%%%%%%%%%%%%%%%%%%%%%%%%%%%%%%%%%%%%%%%%%%%%%%%%%%%%%%%
\section{Proof of Theorem~\ref{thrm:pdg}}
%%%%%%%%%%%%%%%%%%%%%%%%%%%%%%%%%%%%%%%%%%%%%%%%%%%%%%%%%%%%%%%%%%%%%%%%%%%%%%%%

Theorem~\ref{thrm:pdg} states that the automorphisms of a \POP's plan description graph (\PDG) are symmetries of the \POP, that is, if $P$ is a \POP and $\perm \in \auts(\pdg_P)$ then $\tperm$ is a symmetry of $P$.

The proof will be split into two sections.
The first section will show that the automorphisms of a \POP's \PDG are also automorphisms of the \POP's problem description.
Lemmas~\ref{lemma:pdg-init-to}--\ref{lemma:pdg-init} show that $\tperm$ leaves the initial state unchanged, Lemma~\ref{lemma:pdg-goal} shows that it leaves the goal unchanged, and Lemmas~\ref{lemma:pdg-ops-to}--\ref{lemma:pdg-ops} show that it leaves the other operators unchanged.
Lemma~\ref{lemma:pdg-aut} follows directly from these, and shows that the entire problem description is left unchanged.

The second section will show that the automorphisms preserve the validity and well-formed-ness of the \POP.
Lemma~\ref{lemma:pdg-sub-grd} shows that after applying $\tperm$, the \POP{}'s variable binding constraints remain ground and complete \WRT to the operators, and Lemma~\ref{lemma:pdg-po} shows that the transitivity and irreflexivity of the \POP's precedence relation is preserved.
Lemma~\ref{lemma:pdg-valid} shows that $\tperm$ preserves validity.

% automorphism edge
The following observation follows directly from the definition of an automorphism (Definition~\ref{def:automorphism}), and will be used throughout the proofs:

\begin{observation}\label{obs:aut-edge} Let $G = \vertset, \edgeset$ be a graph, and $\perm \in \auts(G)$.
If $\gedge{v_1, v_2} \in \edgeset$ and $\perm(v_1) = v_3$, then $\gedge{v_3, \perm(v_2)} \in \edgeset$.
\end{observation}

The following lemma, which will also be used throughout the proofs, shows that if a function $f$ that maps every $x \in X$ both to and from another element in $X$, then $f$ is a permutation over $X$:

\begin{lemma}\label{lemma:permutation-func} Let $f : S \times S$ be a function and $X \subset S$ be set \ST if $x \in X$ then \myi $f(x) \in X$, and \myii there is a $y \in X$ \ST $f(y) = x$.
Then $f$ is a permutation over $X$.
\end{lemma}

\begin{proof} It suffices to show that $f$ is surjective and injective over $X$. 
Surjectivity requires that for all $x \in X$, there is a $y \in X$ \ST $f(y) = x$, which follows directly from \myii.
Injectivity requires that if $x,y \in X$ and $f(x) = f(y)$ then $x = y$. 
This follows from surjectivity and the pigeonhole principle, and can be shown by \emph{reductio}.
Assume there is an $x,y \in X$ \ST $f(x) = f(y)$ and $x \neq y$.
As every element of $X$ is mapped to another element in $X$ (assumption \myi), and $x$ and $y$ map to the same element, then there must be a $z \in X$ for which there is no $z' \in X$ \ST $f(z') = z$.
As this contradicts surjectivity, $f$ must be injective over $X$.
\end{proof}

%%%%%%%%%%%%%%%%%%%%%%%%%%%%%%%%%%%%%%%%%%%%%%%%%%%%%%%%%%%%%%%%%%%%%%%%%%%%%%%%
% initial state unchanged
The next three lemmas show that the initial state is left unchanged.
Lemma~\ref{lemma:pdg-init-to} shows that if $\perm \in \auts(\pdg_P)$, then $\tperm$ maps every $q(\vec{c}) \in \post(\initoptr)$ to some $q(\vec{d}) \in \post(\initoptr)$:

\begin{lemma}\label{lemma:pdg-init-to} Let $P$ be a \POP, $q(\vec{c}) \in \post(\initoptr)$ and $\perm \in \auts(\pdg_P)$.
Then, there is some $q(\vec{d}) \in \post(\initoptr)$ \ST $\tperm(q(\vec{c})) = q(\vec{d})$.
\end{lemma}

\begin{proof} Let $\vec{c} = \tup{c_1,\ldots,c_n}$.
From Definition~\ref{def:aut-trans}, it suffices to show that there is some $\vec{d} = \tup{d_1,\ldots,d_n}$ \ST $q(\vec{d}) \in \post(\initoptr)$ and for $1 \leq i \leq n$, $\perm(\vvert{c_i}) = \vvert{d_i}$.

From Definition~\ref{def:pdg}, the chain of vertices $\litvert{q(\vec{c})}{1},\ldots,\litvert{q(\vec{c})}{n}$ represents the parameters of $q(\vec{c})$.
First, it is shown that there is some $q(\vec{d}) \in \post(\initoptr)$ \ST for $1 \leq i \leq n$, $\perm(\litvert{q(\vec{c})}{i}) = \litvert{q(\vec{d})}{i}$.
The proof is inductive. 
The base cases show that there is some $q(\vec{d}) \in \post(\initoptr)$ such that $\perm(\litvert{q(\vec{c})}{1}) = \litvert{q(\vec{d})}{1}$ and $\perm(\litvert{q(\vec{c})}{2}) = \litvert{q(\vec{d})}{2}$.
The inductive case shows that as if $\perm(\litvert{q(\vec{c})}{i-1}) = \litvert{q(\vec{d})}{i-1}$ and $\perm(\litvert{q(\vec{c})}{i}) = \litvert{q(\vec{d})}{i}$ hold, then so must $\perm(\litvert{q(\vec{c})}{i+1}) = \litvert{q(\vec{d})}{i+1}$.

First, the base cases are shown for $i = 1$ and $i = 2$.
Let $\litvert{q(\vec{c})}{1} \in \vertset$ be a vertex representing the first parameter of $q(\vec{c})$.
From Definition~\ref{def:automorphism}, $\perm$ must permute $\litvert{q(\vec{c})}{1}$ to a vertex of the same colour, so from Definition~\ref{def:pdg}, $\perm(\litvert{q(\vec{c})}{1}) = \litvert{q(\vec{d})}{i}$, where $\litvert{q(\vec{d})}{i}$ represents the $i$th parameter of some $q(\vec{d}) \in \post(\initoptr) \cup \pre(\goaloptr)$.

Since $\perm(\litvert{q(\vec{c})}{1}) = \litvert{q(\vec{d})}{i}$ and $\gedge{\litvert{q(\vec{c})}{1}, \opvertpost{\initoptr}} \in \edgeset$, it follows from Observation~\ref{obs:aut-edge} that $\gedge{\litvert{q(\vec{d})}{i}, \perm(\opvertpost{\initoptr})} \in \edgeset$.
As $\opvertpost{\initoptr}$ has a unique colour, $\perm(\opvertpost{\initoptr}) = \opvertpost{\initoptr}$ and so $\gedge{\litvert{q(\vec{d})}{i}, \opvertpost{\initoptr}} \in \edgeset$. 
So, from Definition~\ref{def:pdg} it follows that $i = 1$ and $q(\vec{d}) \in \post(\initoptr)$.
Since $\perm(\litvert{q(\vec{c})}{1}) = \litvert{q(\vec{d})}{1}$ and $\gedge{\litvert{q(\vec{c})}{1}, \litvert{q(\vec{c})}{2}} \in \edgeset$, it follows from Observation~\ref{obs:aut-edge} that $\gedge{\litvert{q(\vec{d})}{1}, \perm(\litvert{q(\vec{c})}{2})} \in \edgeset$.
As $\litvert{q(\vec{d})}{1}$ is only linked to $\opvertpost{\initoptr}$ and $\litvert{q(\vec{d})}{2}$, and $\colfunc(\litvert{q(\vec{c})}{2}) \neq \colfunc(\opvertpost{\initoptr})$, it follows that $\perm(\litvert{q(\vec{c})}{2}) = \litvert{q(\vec{d})}{2}$.

Next, the inductive case is shown for $i > 2$.
Let $\litvert{q(\vec{c})}{1},\ldots,\litvert{q(\vec{c})}{n}$ and $\litvert{q(\vec{d})}{1},\ldots,\litvert{q(\vec{d})}{n}$ be two chains of vertices representing $q(\vec{c})$ and $q(\vec{d})$, respectively, let $2 \leq i \leq n$, and let $\perm(\litvert{q(\vec{c})}{i-1}) = \litvert{q(\vec{d})}{i-1}$ and $\perm(\litvert{q(\vec{c})}{i}) = \litvert{q(\vec{d})}{i}$.
Since $\perm(\litvert{q(\vec{c})}{i}) = \litvert{q(\vec{d})}{i}$ and $\gedge{\litvert{q(\vec{c})}{i}, \litvert{q(\vec{c})}{i+1}} \in \edgeset$, then from Observation~\ref{obs:aut-edge}, $\gedge{\litvert{q(\vec{d})}{i}, \perm(\litvert{q(\vec{c})}{i+1})} \in \edgeset$.
As $i > 2$, from Definition~\ref{def:pdg} $\litvert{q(\vec{d})}{i}$ is only connected to $\litvert{q(\vec{d})}{i-1}$ and $\litvert{q(\vec{d})}{i+1}$.
However, as $\perm(\litvert{q(\vec{c})}{i-1}) = \litvert{q(\vec{d})}{i-1}$ and $\litvert{q(\vec{c})}{i-1} \neq \litvert{q(\vec{c})}{i+1}$, it follows that $\perm(\litvert{q(\vec{c})}{i+1}) = \litvert{q(\vec{d})}{i+1}$. 

From Definition~\ref{def:pdg}, for $1 \leq i \leq n$, each parameter vertex $\litvert{q(\vec{c})}{i}$ is connected to the vertex representing that parameter's binding, that is, $\gedge{\litvert{q(\vec{c})}{i}, \vvert{c_i}} \in \edgeset$. 
Let $\vec{d} = \tup{d_1,\ldots,d_n}$.
As $q(\vec{d}) \in \post(\initoptr)$, then similarly, for each $1 \leq i \leq n$, $\gedge{\litvert{q(\vec{d})}{i}, \vvert{d_i}} \in \edgeset$.
From Definition~\ref{def:automorphism}, $\perm$ must map each $\vvert{c_i}$ to a vertex $\gvert$ \ST $\colfunc(v) = \colfunc(\vvert{c_i})$.
And, because $\gedge{\litvert{q(\vec{c})}{i}, \vvert{c_i}} \in \edgeset$ and $\perm(\litvert{q(\vec{c})}{i}) = \litvert{q(\vec{d})}{i}$, it must also hold that $\gedge{\litvert{q(\vec{d})}{i}, \gvert} \in \edgeset$.
From Definition~\ref{def:pdg}, the only vertex that both has the same colour as $\vvert{c_i}$ and is connected to $\litvert{q(\vec{d})}{i}$ is $\vvert{d_i}$, and so it follows that for $\perm(\vvert{c_i}) = \vvert{d_i}$, as desired.
\end{proof}

%%%%%%%%%%%%%%%%%%%%%%%%%%%%%%%%%%%%%%%%%%%%%%%%%%%%%%%%%%%%%%%%%%%%%%%%%%%%%%%%
Lemma~\ref{lemma:pdg-init-from} shows that if $\perm \in \auts(\pdg_P)$ and $q(\vec{c}) \in \post(\initoptr)$, then there is some $q(\vec{d}) \in \post(\initoptr)$ that $\tperm$ maps to $q(\vec{c})$:

\begin{lemma}\label{lemma:pdg-init-from} Let $P$ be a \POP, $q(\vec{c}) \in \post(\initoptr)$ and $\perm \in \auts(\pdg_P)$.
Then, there is some $q(\vec{d}) \in \post(\initoptr)$ \ST $\tperm(q(\vec{d})) = q(\vec{q})$.
\end{lemma}

\begin{proof} The proof is a trivial modification of the proof of Lemma~\ref{lemma:pdg-init-to}. 
Let $\vec{c} = \tup{c_1,\ldots,c_n}$.
From Definition~\ref{def:aut-trans}, it suffices to show that there is some $\vec{d} = \tup{d_1,\ldots,d_n}$ \ST $q(\vec{d}) \in \post(\initoptr)$ and for $1 \leq i \leq n$, $\perm(\vvert{d_i}) = \vvert{c_i}$.
First it is shown inductively that $\perm$ maps the parameter vertices $\litvert{q(\vec{d})}{1},\ldots,\litvert{q(\vec{d})}{n}$ to $\litvert{q(\vec{c})}{1},\ldots,\litvert{q(\vec{c})}{n}$.
The base cases show that for $i = 1$ and $i = 2$, there is some $q(\vec{d}) \in \post(\opset)$ such that $\perm(\litvert{q(\vec{d})}{1}) = \litvert{q(\vec{c})}{1}$ and $\perm(\litvert{q(\vec{d})}{2}) = \litvert{q(\vec{c})}{2}$.
The inductive case shows that for $i >2$, if there is some $q(\vec{d})) \in \post(\opset)$ such that $\perm(\litvert{q(\vec{d})}{i-1}) = \litvert{q(\vec{c})}{i-1}$ and $\perm(\litvert{q(\vec{d})}{i}) = \litvert{q(\vec{c})}{i}$ then $\perm(\litvert{q(\vec{d})}{i+1}) = \litvert{q(\vec{c})}{i+1}$.
Finally, it is shown that for $1 \leq i \leq n$, as $\gedge{\litvert{q(\vec{d})}{i}, \vvert{d_i}}, \gedge{\litvert{q(\vec{c})}{i}, \vvert{c_i}} \in \edgeset$ then $\perm(\vvert{d_i}) = \vvert{c_i}$, as desired.
\end{proof}

%%%%%%%%%%%%%%%%%%%%%%%%%%%%%%%%%%%%%%%%%%%%%%%%%%%%%%%%%%%%%%%%%%%%%%%%%%%%%%%%

Lemma~\ref{lemma:pdg-init} uses the above two results to show that if $\perm \in \auts(\pdg_P)$, then $\tperm$ leaves $\initoptr$ unchanged:

\begin{lemma}\label{lemma:pdg-init} If $P$ is a \POP and $\perm \in \auts(\pdg_P)$ then $\tperm(\initoptr) = \initoptr$.
\end{lemma}

\begin{proof} It suffices to show that $\tperm$ leaves $\post(\initoptr)$ unchanged. 
Lemmas~\ref{lemma:pdg-init-to} and~\ref{lemma:pdg-init-from} show that if $q(\vec{c}) \in \post(\initoptr)$ then $\tperm(q(\vec{c})) \in \post(\initoptr)$ and there is a $q(\vec{d}) \in \post(\initoptr)$ \ST $\tperm(q(\vec{d})) = q(\vec{c})$.
So, from Lemma~\ref{lemma:permutation-func}, $\tperm$ is a permutation over $\post(\initoptr)$, and as $\post(\initoptr)$ is a set, it is unchanged by $\tperm$.
\end{proof}

%%%%%%%%%%%%%%%%%%%%%%%%%%%%%%%%%%%%%%%%%%%%%%%%%%%%%%%%%%%%%%%%%%%%%%%%%%%%%%%%
% goal unchanged
The proofs for Lemmas~\ref{lemma:pdg-init-to}, \ref{lemma:pdg-init-from} and~\ref{lemma:pdg-init} above can be modified to show that if $\perm \in \auts(\pdg_P)$, then $\tperm$ leaves $\goaloptr$ unchanged:

\begin{lemma}\label{lemma:pdg-goal} If $P$ is a \POP and $\perm \in \auts(\pdg_P)$ then $\tperm(\goaloptr) = \goaloptr$.
\end{lemma}

\begin{proof} A trivial modification of the proofs of Lemmas~\ref{lemma:pdg-init-to}, \ref{lemma:pdg-init-from} and~\ref{lemma:pdg-init}. 
\end{proof}

%%%%%%%%%%%%%%%%%%%%%%%%%%%%%%%%%%%%%%%%%%%%%%%%%%%%%%%%%%%%%%%%%%%%%%%%%%%%%%%%
% all other operators unchanged
Lemma~\ref{lemma:pdg-ops-to} shows that if $\perm \in \auts(\pdg_P)$, then $\tperm$ maps every $\optr(\vec{x}) \in \opset$ to some $\optr(\vec{y}) \in \opset$:

\begin{lemma}\label{lemma:pdg-ops-to} Let $P = \tup{\opset, \sub, \precrel}$ be a \POP, $\optr(\vec{x}) \in \opset \setminus \set{\initoptr, \goaloptr}$ and $\perm \in \auts(\pdg_P)$. 
Then, there is some $\optr(\vec{y}) \in \opset \setminus \set{\initoptr, \goaloptr}$ \ST $\name(\optr(\vec{x})) = \name(\optr(\vec{y}))$ and $\tperm(\optr(\vec{x})) = \optr(\vec{y})$.
\end{lemma}

\begin{proof} From Definition~\ref{def:aut-trans}, it suffices to show that for all $\optr(\vec{x}) \in \opset \setminus \set{\initoptr, \goaloptr}$, there is some $\optr(\vec{y}) \in \opset \setminus \set{\initoptr, \goaloptr}$ such that for $1 \leq i \leq \card{\vec{x}}$, $\perm(\vvert{x_i}) = \vvert{y_i}$.
The proof is inductive. 
The base cases show that for all $\optr(\vec{x}) \in \opset \setminus \set{\initoptr, \goaloptr}$, there is some $\optr(\vec{y}) \in \opset \setminus \set{\initoptr, \goaloptr}$ such that $\perm(\vvert{x_1}) = \vvert{y_1}$ and $\perm(\vvert{x_2}) = \vvert{y_2}$.
The inductive case shows that if $\perm(\vvert{x_{i-1}}) = \vvert{y_{i-1}}$ and $\perm(\vvert{x_i}) = \vvert{y_i}$ hold, then so must $\perm(\vvert{x_{i+1}}) = \vvert{y_{i+1}}$.

First, the base cases are shown for $i = 1$ and $i = 2$.
Let $\opvertpre{\optr(\vec{x})}$ and $\opvertpost{\optr(\vec{x})}$ be vertices representing operator $\optr(\vec{x})$'s pre/postcondition sets.
From Definition~\ref{def:pdg}, $\opvertpre{\optr(\vec{x})}$ and $\opvertpost{\optr(\vec{x})}$ must be mapped to vertices $\opvertpre{\optr(\vec{y})}$ and $\opvertpost{\optr(\vec{y})}$ representing the pre/postcondition sets of an operator $\optr(\vec{y})$ \ST $\name(\optr(\vec{x})) = \name(\optr(\vec{y}))$.
Since $\perm(\opvertpre{\optr(\vec{x})}) = \opvertpre{\optr(\vec{y})}$ and $\gedge{\opvertpre{\optr(\vec{x})}, \vvert{x_1}} \in \edgeset$, then from Observation~\ref{obs:aut-edge} it follows that $\gedge{\opvertpre{\optr(\vec{y})}, \perm(\vvert{x_1})} \in \edgeset$.
Further, since $\perm(\opvertpost{\optr(\vec{x})}) = \opvertpost{\optr(\vec{y})}$ and $\gedge{\opvertpost{\optr(\vec{x})}, \vvert{x_1}} \in \edgeset$, then from Observation~\ref{obs:aut-edge} it follows that $\gedge{\opvertpost{\optr(\vec{y})}, \perm(\vvert{x_1})} \in \edgeset$.
Because $\opvertpre{\optr(\vec{y})}$ and $\opvertpost{\optr(\vec{y})}$ are only connected to one vertex, $\vvert{y_1}$, it follows that $\perm(\vvert{x_1}) = \vvert{y_1}$.
Since $\perm(\vvert{x_1}) = \vvert{y_1}$ and $\gedge{\vvert{x_1}, \vvert{x_2}}$, then from Observation~\ref{obs:aut-edge} it follows that $\gedge{\vvert{y_1}, \perm(\vvert{x_2})} \in \edgeset$.
As $\vvert{y_1}$ is only linked to $\opvertpre{\optr(\vec{y})}$, $\opvertpost{\optr(\vec{y})}$ and $\vvert{y_2}$, and $\colfunc(\opvertpre{\optr(\vec{y})}) \neq \colfunc(\vvert{x_2})$ and $\colfunc(\opvertpost{\optr(\vec{y})}) \neq \colfunc(\vvert{x_2})$, it follows that $\perm(\vvert{x_2}) = \vvert{y_2}$.

Next, the inductive case is shown for $i > 2$.
Let $\vvert{x_1},\ldots,\vvert{x_n}$ and $\vvert{y_1},\ldots,\vvert{y_n}$ be two chains of vertices representing $\vec{x}$ and $\vec{y}$, respectively, let $2 \leq i \leq n$, and let $\perm(\vvert{x_{i-1}}) = \vvert{y_{i-1}}$ and $\perm(\vvert{x_i}) = \vvert{y_i}$.
Since $\perm(\vvert{x_i}) = \vvert{y_i}$ and $\gedge{\vvert{x_i}, \vvert{x_{i+1}}} \in \edgeset$, then from Observation~\ref{obs:aut-edge}, $\gedge{\vvert{y_i}, \perm(\vvert{x_{i+1}})} \in \edgeset$.
As $i > 2$, from Definition~\ref{def:pdg} $\vvert{y_i}$ is only connected to $\vvert{y_{i-1}}$ and $\vvert{y_{i+1}}$.
However, as $\perm(\vvert{x_{i-1}}) = \vvert{y_{i-1}}$ and $\vvert{x_{i-1}} \neq \vvert{x_{i-1}}$, it follows that $\perm(\vvert{x_{i+1}}) = \vvert{y_{i+1}}$. 
\end{proof}

%%%%%%%%%%%%%%%%%%%%%%%%%%%%%%%%%%%%%%%%%%%%%%%%%%%%%%%%%%%%%%%%%%%%%%%%%%%%%%%%
Lemma~\ref{lemma:pdg-ops-from} shows that if $\perm \in \auts(\pdg_P)$, and $\optr(\vec{x}) \in \opset$, then there is some $\optr(\vec{y}) \in \opset$ \ST $\name(\optr(\vec{x})) = \name(\optr(\vec{y}))$, and $\tperm(\optr(\vec{y})) = \optr(\vec{x})$:

\begin{lemma}\label{lemma:pdg-ops-from} Let $P = \tup{\opset, \sub, \precrel}$ be a \POP, $\optr(\vec{x}) \in \opset \setminus \set{\initoptr, \goaloptr}$ and $\perm \in \auts(\pdg_P)$. 
Then, there is some $\optr(\vec{y}) \in \opset \setminus \set{\initoptr, \goaloptr}$ \ST $\tperm(\optr(\vec{y})) = \optr(\vec{x})$.
\end{lemma}

\begin{proof} The proof is a trivial modification of the proof of Lemma~\ref{lemma:pdg-ops-to}. 
From Definition~\ref{def:aut-trans}, it suffices to show that for all $\optr(\vec{x}) \in \opset \setminus \set{\initoptr, \goaloptr}$, there is some $\optr(\vec{y}) \in \opset \setminus \set{\initoptr, \goaloptr}$ such that $\name(\optr(\vec{x})) = \name(\optr(\vec{x}))$, and for $1 \leq i \leq \card{\vec{x}}$, $\perm(\vvert{y_i}) = \vvert{x_i}$.
The base cases show that for all $\optr(\vec{x}) \in \opset \setminus \set{\initoptr, \goaloptr}$, there is some $\optr(\vec{y}) \in \opset \setminus \set{\initoptr, \goaloptr}$ such that $\perm(\vvert{y_1}) = \vvert{x_1}$ and $\perm(\vvert{y_2}) = \vvert{x_2}$.
The inductive case shows that if $\perm(\vvert{y_{i-1}}) = \vvert{x_{i-1}}$ and $\perm(\vvert{y_i}) = \vvert{x_i}$ hold, then so must $\perm(\vvert{y_{i+1}}) = \vvert{x_{i+1}}$.
\end{proof}

%%%%%%%%%%%%%%%%%%%%%%%%%%%%%%%%%%%%%%%%%%%%%%%%%%%%%%%%%%%%%%%%%%%%%%%%%%%%%%%%
Lemma~\ref{lemma:pdg-ops} shows that if $\perm \in \auts(\pdg_P)$, then $\tperm$ is an automorphism of $\opset \setminus \set{\initoptr, \goaloptr}$:

\begin{lemma}\label{lemma:pdg-ops} Let $P = \tup{\opset, \sub, \precrel}$ be a \POP and $\perm \in \auts(\pdg_P)$. 
Then, $\tperm(\opset \setminus \set{\initoptr, \goaloptr}) = \opset \setminus \set{\initoptr, \goaloptr}$.
\end{lemma}

\begin{proof} Lemmas~\ref{lemma:pdg-ops-to} and~\ref{lemma:pdg-ops-from} show that if $\optr(\vec{x}) \in \opset \setminus \set{\initoptr, \goaloptr}$ then $\tperm(\optr(\vec{x})) \in \opset \setminus \set{\initoptr, \goaloptr}$, and there is some $\optr(\vec{y}) \in \opset \setminus \set{\initoptr, \goaloptr}$ \ST $\tperm(\optr(\vec{y})) = \optr(\vec{x})$.
Then, from Lemma~\ref{lemma:permutation-func}, $\tperm$ is a permutation over $\opset \setminus \set{\initoptr, \goaloptr}$. 
Thus, as $\opset \setminus \set{\initoptr, \goaloptr}$ is a set, $\tperm(\opset \setminus \set{\initoptr, \goaloptr}) = \opset \setminus \set{\initoptr, \goaloptr}$.
\end{proof}

%%%%%%%%%%%%%%%%%%%%%%%%%%%%%%%%%%%%%%%%%%%%%%%%%%%%%%%%%%%%%%%%%%%%%%%%%%%%%%%%
% prob description is unchanged
Lemma~\ref{lemma:pdg-aut} establishes that if $\perm \in \auts(\pdg_P)$, then $\tperm$ is an automorphism of $\opset$:

\begin{lemma}\label{lemma:pdg-aut}
If $P = \tup{\opset, \sub, \precrel}$ is a \POP and $\perm \in \auts(\pdg_P)$ then $\tperm(\opset) = \opset$.
\end{lemma}	
    
\begin{proof} As $\tperm(\initoptr) = \initoptr$ (Lemma~\ref{lemma:pdg-init}), $\tperm(\goaloptr) = \goaloptr$ (Lemma~\ref{lemma:pdg-goal}) and $\tperm(\opset \setminus \set{\initoptr, \goaloptr}) = \opset \setminus \set{\initoptr, \goaloptr}$ (Lemma~\ref{lemma:pdg-ops}), it follows that $\tperm(\opset) = \opset$.
\end{proof}


%%%%%%%%%%%%%%%%%%%%%%%%%%%%%%%%%%%%%%%%%%%%%%%%%%%%%%%%%%%%%%%%%%%%%%%%%%%%%%%%%

The next section of the proof shows that if $P $ is a \POP and $\perm \in \auts(\pdg_P)$, then applying $\tperm$ to any reinstantiated de/reorder of $P$ preserves its validity and optimality.
The proofs make use of several lemmas.

Lemma~\ref{lemma:aut-vars-consts} shows that if $\perm \in \auts(P)$ then $\tperm$ is a permutation over the variables and constants in $\opset$:

\begin{lemma}\label{lemma:aut-vars-consts} Let $P = \tup{\opset, \sub, \precrel}$ be a \POP and $\perm \in \auts(\pdg_P)$. 
Then, $\tperm(\vars(\opset)) = \vars(\opset)$ and $\tperm(\consts(\opset)) = \consts(\opset)$.
\end{lemma}

\begin{proof} From Definition~\ref{def:pdg}, it follows that if $x \in \vars(\opset)$, then there is some $y \in \vars(\opset)$ \ST $\perm(\vvert{x}) = \vvert{y}$.
Therefore, from Definition~\ref{def:aut-trans}, if $x \in \vars(\opset)$ then $\tperm(x) \in \vars(\opset)$, and there is some $y \in \vars(\opset)$ such that $\tperm(y) = x$.
Thus, from Lemma~\ref{lemma:permutation-func}, $\tperm$ is a permutation over $\vars(\opset)$.
The above proof can be trivially modified to show the same for $\consts(\opset)$.
\end{proof}

Lemma~\ref{lemma:pdg-eq-pres} follows from Definition~\ref{def:aut-trans}, and shows that if $\perm \in \auts(P)$ then $\tperm$ preserves the equality relation over the \POP's variables:

\begin{lemma}\label{lemma:pdg-eq-pres} Let $P$ be a \POP, $Q = \tup{\opset, \sub, \precrel}$ be a reinstantiated reorder of $P$, $\perm \in \auts(\pdg_P)$ and $\tperm(Q) = \tup{\opset, \sub', \precrel'}$. 
Then for all $x, y \in \vars(\opset)$, $\sub(x) = \sub(y)$ \IFF $\sub'(\tperm(x)) = \sub'(\tperm(y))$.
\end{lemma}

\begin{proof} First, it is shown that if $\sub(x) = \sub(y)$ then $\sub'(\tperm(x)) = \sub'(\tperm(y))$. 
If $\sub(x) = \sub(y)$ then there is a $c$ such that $\set{x \substo c, y \substo c} \subseteq \sub$. 
As $\tperm(\sub) = \sub'$, it follows from the definition of a permutation in Section~\ref{sec:permutations} that $\set{\tperm(x) \substo \tperm(c), \tperm(y) \substo \tperm(c)} \subseteq \sub'$, and so $\sub'(\tperm(x)) = \sub'(\tperm(y))$.
Next it is shown that if $\sub(x) \neq \sub(y)$ then $\sub'(\tperm(x)) \neq \sub'(\tperm(y))$.
If $\sub(x) \neq \sub(y)$ then there is a $c \neq d$ such that $\set{x \substo c, y \substo d} \subseteq \sub$. 
As $\tperm(\sub) = \sub'$, it follows that $\set{\tperm(x) \substo \tperm(c), \tperm(y) \substo \tperm(d)} \subseteq \sub'$, and so $\sub'(\tperm(x)) \neq \sub'(\tperm(y))$.
\end{proof}

Lemma~\ref{lemma:pdg-po-obs} follows directly from the definition of a permutation (Section~\ref{sec:permutations}), and makes clear some consequences of applying $\tperm$ to $\precrel$:

\begin{lemma}\label{lemma:pdg-po-obs} Let $P$ be a \POP, $Q = \tup{\opset, \sub, \precrel}$ be a reinstantiated reorder of $P$, $\perm \in \auts(\pdg_P)$ and $\tperm(Q) = \tup{\opset, \sub', \precrel'}$. 
Then:
\begin{itemize}
    \item if $\optr_1 \precrel \optr_2$ \IFF $\tperm(\optr_1) \precrel' \tperm(\optr_2)$, and 
    \item if $\optr_1' \precrel' \optr_2'$ then there is some $\optr_1, \optr_2$ such that $\tperm(\optr_1) = \optr_1'$, $\tperm(\optr_2) = \optr_2'$ and $\optr_1 \prec \optr_2$.
\end{itemize}
\end{lemma}

\begin{proof} Follows directly from the definition of a permutation (Section~\ref{sec:permutations}), which states that a permutation $\perm$ can be applied to a logical structure $\eta$ by simultaneously replacing every term $t \in \eta$ with $\perm(t)$ while maintaining the structure, or type, of $\eta$.
Thus, if $\precrel = \set{(\optr_1, \optr_2)$, $\ldots$, $(\optr_{n-1}, \optr_n)}$ then $\precrel' = \set{(\tperm(\optr_1), \tperm(\optr_2))$,$\ldots$,$(\tperm(\optr_{n-1}), \tperm(\optr_n))}$.
\end{proof}

%%%%%%%%%%%%%%%%%%%%%%%%%%%%%%%%%%%%%%%%%%%%%%%%%%%%%%%%%%%%%%%%%%%%%%%%%%%%%%%%
% bindings well-formed
The next two lemmas show that if $\perm \in \auts(\pdg_P)$, then applying $\tperm$ to a reinstantiated reorder of $P$ results in a well-formed \POP.
First, Lemma~\ref{lemma:pdg-sub-grd} shows that if $\perm \in \auts(\pdg_P)$, then applying $\tperm$ to $\sub$ results in a ground substitution that is complete \WRT to $\opset$:

\begin{lemma}\label{lemma:pdg-sub-grd} Let $P$ be a \POP, $Q = \tup{\opset, \sub, \precrel}$ be a reinstantiated reorder of $P$, $\perm \in \auts(\pdg_P)$ and $\tperm(Q) = \tup{\opset, \sub', \precrel'}$.
Then, $\domain(\sub') = \vars(\opset)$ and for all $x \in \domain(\sub')$, $\sub'(x) \in \consts(\opset)$.
\end{lemma}

\begin{proof} Let $\sub = \set{x_1 \substo c_1, \ldots, x_n \substo c_n}$. 
As $\sub$ is complete \WRT to $\opset$, $\vars(\opset) = \set{x_1,\ldots,x_n}$.
As $\sub' = \tperm(\sub)$, $\sub' = \set{\tperm(x_1) \substo \tperm(c_1), \ldots, \tperm(x_n) \substo \tperm(c_n)}$.
As $\tperm$ is a permutation over $\vars(\opset)$ (Lemma~\ref{lemma:aut-vars-consts}), it follows that $\set{\perm(x_1),$ $\ldots$ $,\perm(x_n)} = \set{x_1,$ $\ldots$ $,x_n} = \vars(\opset)$, and so $\domain(\sub') = \vars(\opset)$.

Since $\set{c_1,\ldots,c_n} \subseteq \consts(\opset)$ and $\tperm$ is a permutation over $\consts(\opset)$ (Lemma~\ref{lemma:aut-vars-consts}), it follows that $\set{\perm(c_1), \ldots ,\perm(c_n)} \subseteq \consts(\opset)$. 
Thus, for all $x \in \vars(\sub')$, $\sub'(x) \in \consts(\opset)$.
\end{proof}

%%%%%%%%%%%%%%%%%%%%%%%%%%%%%%%%%%%%%%%%%%%%%%%%%%%%%%%%%%%%%%%%%%%%%%%%%%%%%%%%
% ordering transitive and acyclic
Lemma~\ref{lemma:pdg-po} shows that if $\perm \in \auts(\pdg_P)$, then applying $\tperm$ to $\precrel$ results in a set of transitively closed, acyclic ordering constraints:

\begin{lemma}\label{lemma:pdg-po} Let $P$ be a \POP, $Q = \tup{\opset, \sub, \precrel}$ be a reinstantiated reorder of $P$, $\perm \in \auts(\pdg_P)$ and $\tperm(Q) = \tup{\opset, \sub', \precrel'}$. 
Then, $\precrel' = \transclos{\precrel'}$ and for all $\optr \in \opset$, $\optr \not\precrel' \optr$.
\end{lemma}

\begin{proof} Irreflexivity is proved by \emph{reductio}.
Assume there is an $\optr' \in \opset$ \ST $\optr' \precrel' \optr'$.
It follows that there must be some $\optr_1$ and $\optr_2$ such that $\optr_1 \precrel \optr_2$ and $\tperm(\optr_1) = \tperm(\optr_2) = \optr'$ (Lemma~\ref{lemma:pdg-po-obs}).
As $\tperm$ is a permutation over $\opset$ (Lemma~\ref{lemma:pdg-aut}), it follows that $\optr_1 = \optr_2$, and so $\optr_1 \precrel \optr_1$ which contradicts the assumption that $\precrel$ is irreflexive.
So, $\precrel'$ must be irreflexive.

Next, transitivity is shown.
Let $\optr_1' \precrel' \optr_2'$, $\optr_2' \precrel' \optr_3'$.
Then there must be an $\optr_1, \optr_2, \optr_3 \in \opset$ such that $\tperm(\optr_1) = \optr_1'$, $\tperm(\optr_2) = \optr_2'$, $\tperm(\optr_3) = \optr_3'$, $\optr_1 \precrel \optr_2$ and $\optr_2 \precrel \optr_3$ (Lemma~\ref{lemma:pdg-po-obs}).
As $\precrel$ is transitive, it follows that $\optr_1 \precrel \optr_3$, and so $\optr_1' \precrel' \optr_3'$ (Lemma~\ref{lemma:pdg-po-obs}), and so $\precrel'$ is transitive.
\end{proof}


%%%%%%%%%%%%%%%%%%%%%%%%%%%%%%%%%%%%%%%%%%%%%%%%%%%%%%%%%%%%%%%%%%%%%%%%%%%%%%%%
% validity unchanged
Lemma~\ref{lemma:pdg-valid} shows if $\perm \in \auts(\pdg_P)$, then applying $\tperm$ to a reinstantiated reorder of $P$ preserves the validity of the \POP.
For the sake of brevity, the proof only demonstrates that \POCL-validity is preserved. 
Proving full \MTC-validity is a trivial but verbose extension that incorporates the possibility of a causal link being re-established by a ``white knight''.

\begin{lemma}\label{lemma:pdg-valid} Let $P$ be a \POP, $Q = \tup{\opset, \sub, \precrel}$ be a reinstantiated reorder of $P$, $\perm \in \auts(\pdg_P)$ and $\tperm(Q) = Q' = \tup{\opset, \sub', \precrel'}$ be the image of $Q$ under $\tperm$. 
Then, $Q$ is valid \IFF $Q'$ is valid.
\end{lemma}

\begin{proof} First it is shown that if $Q$ is \POCL-valid, then $Q'$ is \POCL-valid.
Assume $Q$ is \POCL-valid.
To show that $Q'$ is \POCL-valid, it suffices to show that under $\precrel'$ and $\sub'$, every precondition of every operator is supported by an unthreatened causal link.
Let $\optr_c'$ and $p(\vec{s}')$ be an operator and a literal such that $\consms(\optr_c', p(\vec{s}'))$.
There is therefore an $\optr_c$ such that $\tperm(\optr_c) = \optr_c'$ (Lemma~\ref{lemma:pdg-ops-from}), and therefore a $p(\vec{s}) \in \pre(\optr_c)$ such that $\tperm(\vec{s}) = \vec{s}'$.
As $Q$ is \POCL-valid, there is an $\optr_p$ and $p(\vec{u})$ such that $\prods(\optr_p, p(\vec{u}))$, $\optr_p \precrel \optr_c$ and $\sub(\vec{s}) = \sub(\vec{u})$ (Definition~\ref{def:pocl-valid}).
Let $\tperm(\optr_p) = \optr_p'$.
From Lemma~\ref{lemma:pdg-ops}, there is a $p(\vec{u}') \in \post(\optr_p')$.
As $\optr_p \precrel \optr_c$, it follows that $\optr_p' \precrel' \optr_c'$ (Lemma~\ref{lemma:pdg-po-obs}), and as $\sub(\vec{s}) = \sub(\vec{u})$ it follows that $\sub'(\vec{s}') = \sub'(\vec{u}')$ (Lemma~\ref{lemma:pdg-eq-pres}).
Thus, $\optr_c'$ and $p(\vec{s}')$ are supported by a causal link to $\optr_p'$ and $p(\vec{u}')$.

Let $\optr_t'$ and $p(\vec{v}')$ be an operator and a literal such that $\thrts(\optr_t', p(\vec{v}'))$.
There must be an $\optr_t$ such that $\tperm(\optr_t) = \optr_t'$ (Lemma~\ref{lemma:pdg-ops-from}), and therefore a $p(\vec{v}) \in \post(\optr_t)$ such that $\thrts(\optr_t, p(\vec{v}))$.
As $Q$ is \POCL-valid, it follows that either $\optr_t \precrel \optr_p$, $\optr_c \precrel \optr_t$ or $\sub(\vec{s}) \neq \sub(\vec{v})$ (Definition~\ref{def:pocl-valid}).
If $\optr_t \precrel \optr_p$ then $\optr_t' \precrel' \optr_p'$, if $\optr_c \precrel \optr_t$ then $\optr_c' \precrel' \optr_t'$ (Lemma~\ref{lemma:pdg-po-obs}) and if $\sub(\vec{s}) \neq \sub(\vec{v})$, then $\sub'(\vec{s}') \neq \sub'(\vec{v}')$ (Lemma~\ref{lemma:pdg-eq-pres}).
And so the causal link is unthreatened, and so $Q'$ is \POCL-valid.

Proof that if $Q'$ is \POCL-valid then $Q$ is \POCL-valid is a trivial modification of the above.
If $\optr_c$ and $p(\vec{s})$ are an operator and literal \ST $\consms(\optr_c, p(\vec{s}))$, then, from Lemma~\ref{lemma:pdg-ops-to}, they are mapped by $\tperm$ to some $\optr_c'$ and $p(\vec{s}')$.
As $Q'$ is \POCL-valid, $\optr_c'$ and $p(\vec{s}')$ are supported in $Q'$ by an unthreatened causal link to some $\optr_p'$ and $p(\vec{u}')$ \ST $\prods(\optr_p', p(\vec{u}')$.
Also, from Lemma~\ref{lemma:pdg-ops-to}, $\tperm$ must map some $\optr_p$ and $p(\vec{u})$ to $\optr_p'$ and $p(\vec{u}')$ \ST $\prods(\optr_p, p(\vec{u})$.
And, from Lemmas~\ref{lemma:pdg-po-obs} and~\ref{lemma:pdg-eq-pres}, $\optr_c$ and $p(\vec{s})$ must, in $Q$, be supported by an unthreatened causal link to $\optr_p$ and $p(\vec{u})$.
\end{proof}

%%%%%%%%%%%%%%%%%%%%%%%%%%%%%%%%%%%%%%%%%%%%%%%%%%%%%%%%%%%%%%%%%%%%%%%%%%%%%%%%
% optimality unchanged
Lemma~\ref{lemma:pdg-opt} shows that if $\perm \in \auts(\pdg_P)$, then applying $\tperm$ to $P$ results in an equally optimal \POP, that is, the size of $\precrel$ is unaffected by $\tperm$:

\begin{lemma}\label{lemma:pdg-opt} Let $P$ be a \POP, $Q = \tup{\opset, \sub, \precrel}$ be a reinstantiated reorder of $P$, $\perm \in \auts(\pdg_P)$ and $\tperm(Q) = \tup{\opset, \sub', \precrel'}$. 
Then, $\card{\precrel} = \card{\precrel'}$.
\end{lemma}

\begin{proof} As $\tperm(\precrel) = \precrel'$, for every $(\optr_1, \optr_2) \in \precrel'$ there is some $(\optr_3, \optr_4) \in \precrel$ \ST $\tperm(\optr_3) = \optr_1$ and $\tperm(\optr_4) = \optr_2$.
Therefore, $\precrel'$ cannot have more elements than $\precrel$, that is, $\card{\precrel} \leq \card{\precrel'}$.

Proof that $\card{\precrel} \not< \card{\precrel'}$ is by \emph{reductio}.
If $\card{\precrel} < \card{\precrel'}$, then $\tperm$ must map at least two distinct elements of $\precrel$ to the same element of $\precrel'$.
Assume that there is an $(\optr_1, \optr_2) \in \precrel'$ and $(\optr_3, \optr_4) \neq (\optr_5,\optr_6) \in \precrel$ \ST $\tperm(\optr_3) = \tperm(\optr_5) = \optr_1$ and $\tperm(\optr_4) = \tperm(\optr_6) = \optr_2$.
However, as $\tperm$ is a permutation over $\opset$ (Lemma~\ref{lemma:pdg-aut}), $\tperm(\optr_3) = \tperm(\optr_5)$ \IFF $\optr_3 = \optr_5$ and $\tperm(\optr_4) = \tperm(\optr_6)$ \IFF $\optr_4 = \optr_6$, which contradicts the assumption that $(\optr_3, \optr_4) \neq (\optr_5,\optr_6)$. 
Therefore, $\card{\precrel} \not< \card{\precrel'}$, so $\card{\precrel} = \card{\precrel'}$.
\end{proof}

\pdgthrm*

\begin{proof} Follows from Lemmas~\ref{lemma:pdg-aut}, \ref{lemma:pdg-valid} and~\ref{lemma:pdg-opt}.
\end{proof}
