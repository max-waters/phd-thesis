\begin{figure}[t]
    \centering
    \begin{subfigure}{0.5\textwidth}
    \centering
    \begin{tabular}{c*{3}{p{0.1cm}} }
        0 & 0 & 1 \\
        0 & 1 & 0 \\
        1 & 1 & 0 \\
    \end{tabular}
    \caption{Matrix $\mat{M}_1$}
    \label{fig:double-lex-example-1}
    \end{subfigure}%
    \begin{subfigure}{0.5\textwidth}
    \centering
    \begin{tabular}{c*{3}{p{0.1cm}} }
        0 & 0 & 1 \\
        0 & 1 & 0 \\
        1 & 0 & 1 \\
    \end{tabular}
    \caption{Matrix $\mat{M}_2$}
    \label{fig:double-lex-example-2}
    \end{subfigure} 
    \caption[\dblex failure example]{\dblex failure example. $\mat{M}_2$ is the result of transposing rows $1$ and $2$, and columns $2$ and $3$, and both satisfy \dblex.}
    \label{fig:double-lex-example}
\end{figure}